\documentclass[11pt]{article} % use larger type; default would be 10pt
\usepackage{hyperref}
\usepackage{fullpage}


%%%%%%%%%%%%%%%%%%%%%%%%%%%%%%%%%%%%%%%%%%%%%%%%%%%%%%%%%%%%%%%%%%%%%%%%%%%%%%%%%%%%%%%%%%%%%%%%%%%%%%%%%%%%%%%%%%%%%%%%%%%%%%%%%%%%%%%%%%%%%%%%%%%%%%%%%%%%%%%%%%%%%%%%%%%%%%%%
%% packages
\usepackage[
    type={CC},
    modifier={by-nc-sa},
    version={3.0},
]{doclicense}
\usepackage{authblk}
\usepackage[utf8]{inputenc} % set input encoding (not needed with xelatex)
\usepackage[strict]{changepage}
\usepackage{bold-extra}
    %%% examples of article customizations
    % these packages are optional, depending whether you want the features they provide.
    % see the latex companion or other references for full information.
    
    
    %%% page dimensions
\usepackage{geometry} % to change the page dimensions
\geometry{a4paper} % or letterpaper (us) or a5paper or....
%     \geometry{margin=2in} % for example, change the margins to 2 inches all round
    % \geometry{landscape} % set up the page for landscape
    %   read geometry.pdf for detailed page layout information
 \usepackage{numdef}
   
\usepackage{graphicx} % support the \includegraphics command and options
    % some of the article customisations are relevant for this class
\usepackage{amsmath,amsthm,calligra}
\usepackage{amsfonts} % math fonts such as \mathbb{}
\usepackage{amssymb} % \therefore
% \usepackage{bickham}
\usepackage{cryptocode}
\usepackage{framed} 
    % \usepackage[parfill]{parskip} % activate to begin paragraphs with an empty line rather than an indent
    
    %%% packages
\usepackage{booktabs} % for much better looking tables
\usepackage{array} % for better arrays (eg matrices) in maths
\usepackage{paralist} % very flexible & customisable lists (eg. Enumerate/itemize, etc.)
\usepackage{verbatim} % adds environment for commenting out blocks of text & for better verbatim
\usepackage{subfig} % make it possible to include more than one captioned figure/table in a single float
    % % these packages are all incorporated in the memoir class to one degree or another...
\usepackage{mathrsfs}
\usepackage{booktabs}
\usepackage{makecell}
\usepackage{adjustbox}
% \usepackage{pzccal}
% \DeclareFontFamily{OT1}{pzc}{}
% \DeclareFontShape{OT1}{pzc}{m}{it}{<-> s * [1.10] pzcmi7t}{}
% \DeclareMathAlphabet{\mathpzc}{OT1}{pzc}{m}{it}
\usepackage{pgfplots}

\renewcommand\theadalign{bc}
\renewcommand\theadfont{\bfseries}
\renewcommand\theadgape{\Gape[4pt]}
  
    %%% headers & footers
\usepackage{fancyhdr} % this should be set after setting up the page geometry
\pagestyle{fancy} % options: empty , plain , fancy
\renewcommand{\headrulewidth}{0pt} % customise the layout...
\lhead{}\chead{}\rhead{}
\lfoot{}\cfoot{\thepage}\rfoot{}
    
    
    %%% section title appearance
\usepackage{sectsty}
\allsectionsfont{\sffamily\mdseries\upshape} % (see the fntguide.pdf for font help)
    % (this matches context defaults)
    
    %%% toc (table of contents) appearance
\usepackage[nottoc,notlof,notlot]{tocbibind} % put the bibliography in the toc
\usepackage[titles,subfigure]{tocloft} % alter the style of the table of contents
\renewcommand{\cftsecfont}{\rmfamily\mdseries\upshape}
\renewcommand{\cftsecpagefont}{\rmfamily\mdseries\upshape} % no bold!
\usepackage[scr=esstix]  % heavily sloped
%             cal=esstix]   % slightly sloped
           {mathalpha}
 %%% end article customizations
 
 %%%%%%%%%%%%%%%%%%%%%%%%%%%%%%%%%%%%%%%%%%%%%%%%%%%%%%%%%%%%%%%%%%%%%%%%%%%%%%%%%%%%%%%%%%%%%%%%%%%%%%%%%%%%%%%%%%%%%%%%%%%%%%%%%%%%%%%%%%%%%%%%%%%%%%%%%%%%%%%%%%%%%%%%%%%%%%%%
 % macros
 \newcommand{\code}[1]{\texttt{#1}}
\newcommand\tstrut{\rule{0pt}{2.6ex}}         % = `top' strut
\newcommand\bstrut{\rule[-0.9ex]{0pt}{0pt}}   % = `bottom' strut
\newcommand{\bgamma}{\boldsymbol{\gamma}}
\newcommand{\bsigma}{\boldsymbol{\sigma}}
\newcommand{\plonk}{\ensuremath{\mathcal{P} \mathfrak{lon}\mathcal{K}}\xspace}
\newcommand{\cq}{\ensuremath{\mathpgoth{cq} }\xspace}
\newcommand{\cqlin}{\ensuremath{\mathpgoth{cq}\mathscr{\text{\calligra{lin}}} }\xspace}
\newcommand{\cqstar}{\ensuremath{\mathpgoth{cq^{\mathbf{*}} }}\xspace}
\newcommand{\protogal}{{\scshape Proto\bfseries{galaxy}}\xspace}
\newcommand{\stackproofs}{s{\ensuremath{\mathfrak{t} \mathpgoth {ack}}\bfseries{proofs}}\xspace}
\newcommand{\protostar}{{\scshape Protostar}\xspace}
\newcommand{\hypernova}{{HyperNova}\xspace}
\newcommand{\flookup}{\ensuremath{\mathsf{\mathpgoth{Flookup}}}\xspace}
\newcommand{\baloo}{\ensuremath{\mathrm{ba}\mathit{loo}}\xspace}
\newcommand{\caulkp}{\ensuremath{\mathsf{\mathrel{Caulk}\mathrel{\scriptstyle{+}}}}\xspace}
\newcommand{\caulk}{\ensuremath{\mathsf{Caulk}}\xspace}
\newcommand{\plookup}{\ensuremath{\mathpgoth{plookup}}\xspace}
\newcommand{\tablegroup}{\ensuremath{\mathbb{H}}\xspace}
\newcommand{\V}{\ensuremath{\mathbf{V} }\xspace}
\newcommand{\relbase}{\ensuremath{\rel_0}\xspace}
\mathchardef\mhyphen="2D 

\newcommand{\instbase}{\ensuremath{\inst_0}\xspace}
\newcommand{\witbase}{\ensuremath{\wit_0}\xspace}


\newcommand{\papertitle}{\stackproofs: Private proofs of stack and contract execution using \protogal}
%\newcommand{\authorname}}
\newcommand{\company}{}
\title{ \papertitle \\[0.72cm]}
% \author{}
% \author{ Liam Eagen \\ \tt{Alpen Labs}   \and Ariel Gabizon \\ \tt{Zeta Function Technologies} } 
\author[1]{Liam Eagen}
\author[2]{Ariel Gabizon}
\author[3]{Marek Sefranek}
\author[2]{Patrick Towa}
\author[2]{Zachary J. Williamson}
\affil[1]{Alpen Labs}
\affil[2]{Aztec Labs}
\affil[3]{TU Wien}
% 
% 	\large{\authorname} \\[0.5cm] \large{\company}
% 	\\ {DRAFT}
%}
%\date{} % activate to display a given date or no date (if empty),

% otherwise the current date is printed 
\DeclareMathAlphabet{\mathpgoth}{OT1}{pgoth}{m}{n}	
\ProvidesPackage{numdef}



%% Ariel Macros:
% \num\newcommand{\G1}{\ensuremath{{\mathbb G}_1}\xspace}
\newcommand{\Gi}{\ensuremath{{\mathbb G}_i}\xspace}
\newcommand{\G}{\ensuremath{{\mathbb G}}\xspace}
\newcommand{\Gstar}{\ensuremath{{\mathbb G}^*}\xspace}
\newcommand{\x}{\ensuremath{\mathbf{x}}\xspace}
\newcommand{\z}{\ensuremath{\mathbf{z}}\xspace}
\newcommand{\X}{\ensuremath{\mathbf{X}}\xspace}

% \num\newcommand{\G2}{\ensuremath{{\mathbb G}_2}\xspace}
%\num\newcommand{\G11}{\ensuremath{\G1\setminus \set{0} }\xspace}
%\num\newcommand{\G21}{\ensuremath{\G2\setminus \set{0} }\xspace}
\newcommand{\grouppair}{\ensuremath{G^*}\xspace}
\newcommand{\prvperm}{\ensuremath{\mathrm{P_{\mathsf{\sigma}}}}\xspace}
\newcommand{\verperm}{\ensuremath{\mathrm{V_{\mathsf{\sigma}}}}\xspace}
\newcommand{\alg}{\ensuremath{\mathscr{A}}\xspace}

\newcommand{\Gt}{\ensuremath{{\mathbb G}_t}\xspace}
\newcommand{\F}{\ensuremath{\mathbb F}\xspace}
\newcommand{\Fstar}{\ensuremath{\mathbb F^*}\xspace}

\newcommand{\help}[1]{$#1$-helper\xspace}
\newcommand{\randompair}[1]{\ensuremath{\mathsf{randomPair}(#1)}\xspace}
\newcommand{\pair}[1]{$#1$-pair\xspace}
\newcommand{\pairs}[1]{$#1$-pairs\xspace}
\newcommand{\chalpoint}{\ensuremath{\mathfrak{z}}\xspace}

% \newcommand{\pairone}[1]{\G1-$#1$-pair\xspace}
% \newcommand{\pairtwo}[1]{\G2-$#1$-pair\xspace}
% \newcommand{\sameratio}[2]{\ensuremath{\mathsf{SameRatio}(#1,#2)}\xspace}
\newcommand{\vecc}[2]{\ensuremath{\left(#1\right)_{#2}}\xspace}
\newcommand{\players}{\ensuremath{[n]}\xspace}
\newcommand{\adv}{\ensuremath{\mathcal A}\xspace}
\newcommand{\advprime}{\ensuremath{\mathcal{A}'}\xspace}
\newcommand{\extprime}{\ensuremath{E'}\xspace}
\newcommand{\advrand}{\ensuremath{\mathsf{rand}_{\adv}}\xspace}
% \num\newcommand{\srs1}{\ensuremath{\mathsf{srs_1}}\xspace}
% \num\newcommand{\srs2}{\ensuremath{\mathsf{srs_2}}\xspace}
\newcommand{\srs}{\ensuremath{\mathsf{srs}}\xspace}
\newcommand{\regsrs}[1]{\ensuremath{\sett{\enc1{x^i}}{i\in \set{0,\ldots,#1-1}}}\xspace}
\newcommand{\srsm}{\ensuremath{\mathsf{srs}_M}\xspace}
\newcommand{\srsext}{\ensuremath{\mathsf{srs^*}}\xspace}
\newcommand{\srsbase}{\ensuremath{\mathsf{srs_0}}\xspace}
\newcommand{\srsi}{\ensuremath{\mathsf{srs_i}}\xspace}
\newcommand{\com}{\ensuremath{\mathsf{com}}\xspace}
\newcommand{\comperm}{\ensuremath{\mathsf{com_{\sigma}}}\xspace}
\newcommand{\cm}{\ensuremath{\mathsf{cm}}\xspace}
\newcommand{\cmsig}{\ensuremath{\mathsf{cm_\sigma}}\xspace}
\newcommand{\open}{\ensuremath{\mathsf{open}}\xspace}
\newcommand{\openperm}{\ensuremath{\mathsf{open_{\sigma}}}\xspace}
\newcommand{\sigof}[1]{\ensuremath{\sigma(#1)}\xspace}
\newcommand{\proverexp}{\ensuremath{\mathsf{e}}\xspace}
\newcommand{\reducedelems}{\ensuremath{\mathsf{r}}\xspace}

\newcommand{\ci}{\ensuremath{\mathrm{CI}}\xspace}
\renewcommand{\deg}{\ensuremath{\mathrm{deg}}\xspace}
\newcommand{\pairvec}[1]{$#1$-vector\xspace}
\newcommand{\Fq}{\ensuremath{\mathbb{F}_q}\xspace}
\newcommand{\randpair}[1]{\ensuremath{\mathsf{rp}_{#1}}\xspace}
\newcommand{\randpairone}[1]{\ensuremath{\mathsf{rp}_{#1}^{1}}\xspace}
\newcommand{\negl}{\ensuremath{\mathsf{negl}(\lambda)}\xspace}
\newcommand{\randpairtwo}[1]{\ensuremath{\mathsf{rp_{#1}^2}}\xspace}%the randpair in G2
% \newcommand{\nilp}{\ensuremath{\mathscr N}\xspace}
% \newcommand{\groupgen}{\ensuremath{\mathscr G}\xspace}
% \newcommand{\qap}{\ensuremath{\mathscr Q}\xspace}
\newcommand{\polprot}[4]{$(#1,#2,#3,#4)$-polynomial protocol}
\newcommand{\rangedprot}[5]{$#5$-ranged $(#1,#2,#3,#4)$-polynomial protocol}

\newcommand{\rej}{\ensuremath{\mathsf{reject}}\xspace}
\newcommand{\acc}{\ensuremath{\mathsf{accept}}\xspace}
\newcommand{\accept}{\ensuremath{\mathsf{accept}}\xspace}
\newcommand{\res}{\ensuremath{\mathsf{res}}\xspace}
% \newcommand{\sha}[1]{\ensuremath{\mathsf{COMMIT}(#1)}\xspace}
%  \newcommand{\shaa}{\ensuremath{\mathsf{COMMIT}}\xspace}
 \newcommand{\comm}[1]{\ensuremath{\enc1{#1(x)}}\xspace}
 \newcommand{\defeq}{:=}

\newcommand{\B}{\ensuremath{\set{0,1}}\xspace}
\newcommand{\dom}{\ensuremath{H}\xspace}
\newcommand{\C}{\ensuremath{\vec{C}}\xspace}
\newcommand{\Btwo}{\ensuremath{\vec{B_2}}\xspace}
\newcommand{\treevecsimp}{\ensuremath{(\tau,\rho_A,\rho_A \rho_B,\rho_A\alpha_A,\rho_A\rho_B\alpha_B, \rho_A\rho_B\alpha_C,\beta,\beta\gamma)}\xspace}% The sets of elements used in simplifed relation tree in main text body
\newcommand{\rcptc}{random-coefficient subprotocol\xspace}
\newcommand{\rcptcparams}[2]{\ensuremath{\mathrm{RCPC}(#1,#2)}\xspace}
\newcommand{\verifyrcptcparams}[2]{\ensuremath{\mathrm{\mathsf{verify}RCPC}(#1,#2)}\xspace}
\newcommand{\randadv}{\ensuremath{\mathsf{rand}_{\adv}}\xspace}
 \num\newcommand{\ex1}[1]{\ensuremath{#1\cdot g_1}\xspace}
 \num\newcommand{\ex2}[1]{\ensuremath{#1\cdot g_2}\xspace}
 \newcommand{\pr}{\mathrm{Pr}}
%  \newcommand{\powervec}[2]{\ensuremath{(1,#1,#1^{2},\ldots,#1^{#2})}\xspace}
%  \newcommand{\partition}{\ensuremath{\mathcal{T}}\xspace}
%  \newcommand{\partof}[1]{\ensuremath{{\partition_{#1}}}\xspace}
% \num\newcommand{\out1}[1]{\ensuremath{\ex1{\powervec{#1}{d}}}\xspace}
% \num\newcommand{\out2}[1]{\ensuremath{\ex2{\powervec{#1}{d}}}\xspace}
%  \newcommand{\nizk}[2]{\ensuremath{\mathrm{NIZK}(#1,#2)}\xspace}% #2 is the hash concatenation input
%  \newcommand{\verifynizk}[3]{\ensuremath{\mathrm{VERIFY\mhyphen NIZK}(#1,#2,#3)}\xspace}
\newcommand{\protver}{protocol verifier\xspace} 
\newcommand{\hash}{\ensuremath{\mathcal{H}}\xspace} 
\newcommand{\mulgroup}{\ensuremath{\F^*}\xspace}
\newcommand{\lag}[1]{\ensuremath{L_{#1}}\xspace} 
\newcommand{\sett}[2]{\ensuremath{\set{#1}_{#2}}\xspace}
\newcommand{\omegaprod}{\ensuremath{\alpha_{\omega}}\xspace}
\newcommand{\lagvec}[1]{\ensuremath{\mathrm{LAG}_{#1}}\xspace}
\newcommand{\trapdoor}{\ensuremath{r}}
\newcommand{\trapdoorext}{\ensuremath{r_{\mathrm{ext}}}\xspace}
% \newcommand{\trapdoorsim}{\ensuremath{r_{\mathrm{sim}}}\xspace}
\renewcommand{\mod}{\ensuremath{\;\mathrm{mod}\;}}
\newcommand{\hsub}{\ensuremath{H^*}\xspace}
\num\newcommand{\enc1}[1]{\ensuremath{\left[#1\right]_1}\xspace}
\newcommand{\enci}[1]{\ensuremath{\left[#1\right]_i}\xspace}
\num\newcommand{\enc2}[1]{\ensuremath{\left[#1\right]_2}\xspace}
\newcommand{\gen}{\ensuremath{\mathsf{gen}}\xspace}
\newcommand{\hgen}{\ensuremath{\omega}\xspace}
\newcommand{\gops}{\G1-operations\xspace}
\newcommand{\nlogngops}{$O(n\log n)$ \G1-operations\xspace}
\newcommand{\nlognfops}{$O(n\log n)$ \F-operations\xspace}
\newcommand{\fops}{\F-operations\xspace}
\newcommand{\prv}{\ensuremath{\mathsf{\mathbf{P}}}\xspace}
\newcommand{\prvpoly}{\ensuremath{\mathrm{P_{\mathsf{poly}}}}\xspace}
\newcommand{\prvpc}{\ensuremath{\mathrm{P_{\mathsf{PC}}}}\xspace}
\newcommand{\verpoly}{\ensuremath{\mathrm{V_{\mathsf{poly}}}}\xspace}
\newcommand{\verpc}{\ensuremath{\mathrm{V_{\mathsf{PC}}}}\xspace}
\newcommand{\ideal}{\ensuremath{\mathcal{I}}\xspace}
\newcommand{\prf}{\ensuremath{\mathsf{\pi}}\xspace}
\newcommand{\prfone}{\ensuremath{\mathsf{\pi_1}}\xspace}
% \newcommand{\simprv}{\ensuremath{\mathrm{P^{sim}}}\xspace}

%\newcommand{\enc}[1]{\ensuremath{\left[#1\right]}\xspace}
%\num\newcommand{\G0}{\ensuremath{\mathbf{G}}\xspace}
\newcommand{\GG}{\ensuremath{\mathbf{G^*}}\xspace}  % would have liked to call this G01 but problem with name
\num\newcommand{\g0}{\ensuremath{\mathbf{g}}\xspace}
\newcommand{\inst}{\ensuremath{\phi}\xspace}
\newcommand{\newinst}{\ensuremath{\phi^*}\xspace}
\newcommand{\row}{\ensuremath{\mathsf{R}}\xspace}
\newcommand{\col}{\ensuremath{\mathsf{C}}\xspace}
\newcommand{\inp}{\ensuremath{\mathsf{x}}\xspace}
\newcommand{\wit}{\ensuremath{\mathsf{\omega}}\xspace}
\newcommand{\eps}{\ensuremath{\varepsilon}\xspace}
\newcommand{\inpF}{\ensuremath{\mathscr{x}}\xspace}
\newcommand{\witF}{\ensuremath{\mathscr{w}}\xspace}
\newcommand{\instFprime}{\ensuremath{\mathpgoth{x}}\xspace}
\newcommand{\witFprime}{\ensuremath{\mathpgoth{w}}\xspace}
\newcommand{\acchash}{\ensuremath{\mathscr{h}}\xspace}
\newcommand{\accnew}{\ensuremath{\mathscr{acc}}\xspace}
\newcommand{\cnt}{\ensuremath{\mathsf{count}}\xspace}
\newcommand{\ver}{\ensuremath{\mathsf{\mathbf{V}}}\xspace}
\newcommand{\verpg}{\ensuremath{\ver_{PG}}\xspace}
\newcommand{\per}{\ensuremath{\mathsf{\mathbf{P}}}\xspace}
\newcommand{\perpg}{\ensuremath{\per_{PG}}\xspace}
\newcommand{\sonic}{\ensuremath{\mathsf{Sonic}}\xspace}
\newcommand{\aurora}{\ensuremath{\mathsf{Aurora}}\xspace}
\newcommand{\auroralight}{\ensuremath{\mathsf{Auroralight}}\xspace}
\newcommand{\groth}{\ensuremath{\mathsf{Groth'16}}\xspace}
\newcommand{\kate}{\ensuremath{\mathsf{KZG}}\xspace}
\newcommand{\rel}{\ensuremath{\mathcal{R}}\xspace}
\newcommand{\relrand}{\ensuremath{\mathcal{R^{\mathsf{rand}}}}\xspace}
\newcommand{\lang}{\ensuremath{\mathcal{L}}\xspace}
\newcommand{\params}{\ensuremath{\mathsf{params}_{\inst}}\xspace}
\newcommand{\protparams}{\ensuremath{\mathsf{params}_{\inst}^\advv}\xspace}
\num\newcommand{\p1}{\ensuremath{P_1}\xspace}
\newcommand{\advv}{\ensuremath{\mathcal{A}^{\mathbf{*}}}\xspace} % the adversary that uses protocol adversary as black box
\newcommand{\crs}{\ensuremath{\sigma}\xspace}
%\num\newcommand{\crs1}{\ensuremath{\mathrm{\sigma}_1}\xspace}
%\num\newcommand{\crs2}{\ensuremath{\mathrm{\sigma}_2}\xspace}
\newcommand{\set}[1]{\ensuremath{\left\{#1\right\}}\xspace}
% \newcommand{\hgen}{\ensuremath{\mathbf{\omega}}\xspace}
\newcommand{\vgen}{\ensuremath{\mathbf{g}}\xspace}
% \renewcommand{\sim}{\ensuremath{\mathsf{sim}}\xspace}%the distribution of messages when \advv simulates message of \p1
\newcommand{\real}{\ensuremath{\mathsf{real}}\xspace}%the distribution of messages when \p1 is honest and \adv controls rest of players
 \newcommand{\koevec}[2]{\ensuremath{(1,#1,\ldots,#1^{#2},\alpha,\alpha #1,\ldots,\alpha #1^{#2})}\xspace}
\newcommand{\mida}{\ensuremath{A_{\mathrm{mid}}}\xspace}
\newcommand{\midb}{\ensuremath{B_{\mathrm{mid}}}\xspace}
\newcommand{\midc}{\ensuremath{C_{\mathrm{mid}}}\xspace}
\newcommand{\chal}{\ensuremath{\mathsf{challenge}}\xspace}
\newcommand{\attackparams}{\ensuremath{\mathsf{params^{pin}}}\xspace}
\newcommand{\pk}{\ensuremath{\mathsf{pk}}\xspace}
\newcommand{\attackdist}[2]{\ensuremath{AD_{#1}}\xspace}
% \renewcommand{\neg}{\ensuremath{\mathsf{negl}(\lambda)}\xspace}
\newcommand{\ro}{\ensuremath{{\mathscr R}}\xspace}
\newcommand{\elements}[1]{\ensuremath{\mathsf{elements}_{#1}}\xspace}
 \num\newcommand{\elmpowers1}[1]{\ensuremath{\mathrm{\mathsf{e}}^1_{#1}}\xspace}
 \num\newcommand{\elmpowers2}[1]{\ensuremath{\mathrm{\mathsf{e}}^2_{#1}}\xspace}
\newcommand{\elempowrs}[1]{\ensuremath{\mathsf{e}_{#1}}\xspace}
 \newcommand{\secrets}{\ensuremath{\mathsf{secrets}}\xspace}
 \newcommand{\polysofdeg}[1]{\ensuremath{\F_{< #1}[X]}\xspace}
 \newcommand{\polysofdegeq}[1]{\ensuremath{\F_{\leq #1}[X]}\xspace}
 \newcommand{\pols}{\ensuremath{\F[X]}\xspace}
 \newcommand{\bivar}[1]{\ensuremath{\F_{< #1}[X,Y]}\xspace}
 \newcommand{\sig}{\ensuremath{\mathscr{S}}\xspace}
 \newcommand{\prot}{\ensuremath{\mathscr{P}}\xspace}
 \newcommand{\protstar}{\ensuremath{\mathscr{P}^*}\xspace}
 \newcommand{\PCscheme}{\ensuremath{\mathscr{S}}\xspace}
 \newcommand{\protprime}{\ensuremath{\mathscr{P^*}}\xspace}
 \newcommand{\sigprv}{\ensuremath{\mathsf{P_{sc}}}\xspace}
 \newcommand{\sigver}{\ensuremath{\mathsf{V_{sc}}}\xspace}
 \newcommand{\sigpoly}{\ensuremath{\mathsf{S_{\sigma}}}\xspace}
 \newcommand{\idpoly}{\ensuremath{\mathsf{S_{ID}}}\xspace}
\newcommand{\idpolyevala}{\ensuremath{\mathsf{\bar{s}_{ID1}}}\xspace}
\newcommand{\sigpolyevala}{\ensuremath{\mathsf{\bar{s}_{\sigma1}}}\xspace}
\newcommand{\sigpolyevalb}{\ensuremath{\mathsf{\bar{s}_{\sigma2}}}\xspace}
\newcommand{\bctv}{\ensuremath{\mathsf{BCTV}}\xspace}
\newcommand{\PI}{\ensuremath{\mathsf{PI}}\xspace}
\newcommand{\PIb}{\ensuremath{\mathsf{PI_B}}\xspace}
\newcommand{\PIc}{\ensuremath{\mathsf{PI_C}}\xspace}
\newcommand{\dl}[1]{\ensuremath{\widehat{#1}}\xspace}
\newcommand{\obgen}{\ensuremath{\mathcal O}\xspace}
\newcommand{\PC}{\ensuremath{\mathscr{P}}\xspace}
\newcommand{\permscheme}{\ensuremath{\sigma_\mathscr{P}}\xspace}

	
\newcommand{\selleft}{\ensuremath{\mathbf{q_L}}\xspace}
\newcommand{\selright}{\ensuremath{\mathbf{q_R}}\xspace}
\newcommand{\selout}{\ensuremath{\mathbf{q_O}}\xspace}
\newcommand{\selmult}{\ensuremath{\mathbf{q_M}}\xspace}
\newcommand{\selconst}{\ensuremath{\mathbf{q_C}}\xspace}
\newcommand{\selectors}{\ensuremath{\mathcal{Q}}\xspace}
\newcommand{\lvar}{\ensuremath{\mathbf{a}}\xspace}
\newcommand{\vars}{\ensuremath{\mathcal{V}}\xspace}
\newcommand{\rvar}{\ensuremath{\mathbf{b}}\xspace}
\newcommand{\ovar}{\ensuremath{\mathbf{c}}\xspace}
\newcommand{\pubvars}{\ensuremath{\mathcal{I}}\xspace}
\newcommand{\assignment}{\ensuremath{\mathbf{x}}\xspace}
\newcommand{\constsystem}{\ensuremath{\mathscr{C}}\xspace}
\newcommand{\relof}[1]{\ensuremath{\rel_{#1}}\xspace}
\newcommand{\pubinppoly}{\ensuremath{\mathsf{PI}}\xspace}
\newcommand{\sumi}[1]{\sum_{i\in[#1]}}
\newcommand{\sumzertok}[1]{\sum_{#1=0}^{k}}
\newcommand{\sumpoly}[1]{\sum_{i=0}^{#1-1}}
\newcommand{\summ}[1]{\sum_{i\in[#1]}}
\newcommand{\sumj}[1]{\sum_{j\in[#1]}}
\newcommand{\ZeroH}{\ensuremath{Z_H} \xspace}
\newcommand{\lpoly}{\ensuremath{\mathsf{a}}\xspace}
\newcommand{\rpoly}{\ensuremath{\mathsf{b}}\xspace}
\newcommand{\opoly}{\ensuremath{\mathsf{c}}\xspace}
\newcommand{\idpermpoly}{\ensuremath{\mathsf{z}}\xspace}
\newcommand{\lagrangepoly}{\ensuremath{\mathsf{L}}\xspace}
\newcommand{\zeropoly}{\ensuremath{\mathsf{\ZeroH}}\xspace}
\newcommand{\selmultpoly}{\ensuremath{\mathsf{q_M}}\xspace}
\newcommand{\selleftpoly}{\ensuremath{\mathsf{q_L}}\xspace}
\newcommand{\selrightpoly}{\ensuremath{\mathsf{q_R}}\xspace}
\newcommand{\seloutpoly}{\ensuremath{\mathsf{q_O}}\xspace}
\newcommand{\selconstpoly}{\ensuremath{\mathsf{q_C}}\xspace}
\newcommand{\idcomm}{\ensuremath{[s_{\mathsf{ID1}}]_1}\xspace}
\newcommand{\sigcomma}{\ensuremath{[s_{\mathsf{\sigma1}}]_1}\xspace}
\newcommand{\sigcommb}{\ensuremath{[s_{\mathsf{\sigma2}}]_1}\xspace}
\newcommand{\sigcommc}{\ensuremath{[s_{\mathsf{\sigma3}}]_1}\xspace}
\newcommand{\selleftcomm}{\ensuremath{[q_\mathsf{L}]_1}\xspace}
\newcommand{\selrightcomm}{\ensuremath{[q_\mathsf{R}]_1}\xspace}
\newcommand{\seloutcomm}{\ensuremath{[q_\mathsf{O}]_1}\xspace}
\newcommand{\selconstcomm}{\ensuremath{[q_\mathsf{C}]_1}\xspace}
\newcommand{\selmultcomm}{\ensuremath{[q_\mathsf{M}]_1}\xspace}

\newcommand{\multlinecomment}[1]{\directlua{-- #1}}
    

\newtheorem{lemma}{Lemma}[section]
\newtheorem{thm}[lemma]{Theorem}
\newtheorem{dfn}[lemma]{Definition}
\newtheorem{remark}[lemma]{Remark}

\newtheorem{claim}[lemma]{Claim}
\newtheorem{corollary}[lemma]{Corollary}
\newcommand{\Z}{\mathbb{Z}}
\newcommand{\R}{\mathcal{R}}
\newcommand{\crct}{\ensuremath{\mathsf{C}}\xspace}
\newcommand{\A}{\ensuremath{\mathcal{A}}\xspace}
%\newcommand{\G}{\mathcal{G}}
\newcommand{\Gr}{\mathbb{G}}
%\newcommand{\com}{\textsf{com}}  Ariel defined equivalent that also works in math mode
\newcommand{\cgen}{\text{cgen}}
\newcommand{\poly}{\ensuremath{\mathsf{poly(\lambda)}}\xspace}
\newcommand{\snark}{\ensuremath{\mathsf{snark}}\xspace}
\newcommand{\grandprod}{\mathsf{prod}}
\newcommand{\perm}{\mathsf{S}}
%\newcommand{\open}{\mathsf{open}}
\newcommand{\update}{\mathsf{update}}
\newcommand{\Prove}{\mathcal{P}}
\newcommand{\Verify}{\mathcal{V}}
\newcommand{\Extract}{\mathcal{E}}
\newcommand{\Simulate}{\mathcal{S}}
\newcommand{\Unique}{\mathcal{U}}
\newcommand{\Rpoly}{\R{\poly}}
\newcommand{\Ppoly}{\Prove{\poly}}
\newcommand{\Vpoly}{\Verify{\poly}}
\newcommand{\Psnark}{\prv}%{\Prove{\snark}}
\newcommand{\Vsnark}{\ver}%{\Verify{\snark}}
\newcommand{\Rprod}{\R{\grandprod}}
\newcommand{\Pprod}{\Prove{\grandprod}}
\newcommand{\Vprod}{\Verify{\grandprod}}
\newcommand{\Rperm}{\R{\perm}}
\newcommand{\Pperm}{\Prove{\perm}}
\newcommand{\Vperm}{\Verify{\perm}}
% \newcommand{\zw}[1]{{\textcolor{magenta}{Zac:#1}}}
\newcommand{\ag}[1]{{\textcolor{blue}{\emph{Ariel:#1}}}}
\newcommand{\prob}{\ensuremath{\mathrm{Pr}}\xspace}
\newcommand{\extprot}{\ensuremath{E_{\prot}}\xspace}
\newcommand{\transcript}{\ensuremath{\mathsf{transcript}}\xspace}
\newcommand{\extpc}{\ensuremath{E_{\PCscheme}}\xspace}
\newcommand{\advpc}{\ensuremath{\mathcal A_{\PCscheme}}\xspace}
\newcommand{\advprot}{\ensuremath{\mathcal A_{\prot}}\xspace}
\newcommand{\protmany}{\ensuremath{\prot_k}\xspace}

\usepackage{pifont}% http://ctan.org/pkg/pifont
\newcommand{\cmark}{\ding{51}}%
\newcommand{\xmark}{\ding{55}}%
\newcommand{\marlin}{\ensuremath{\mathsf{Marlin}}\xspace}
\newcommand{\fractal}{\ensuremath{\mathsf{Fractal}}\xspace}
% \newcommand{\Rsnark}{\R{\snark}}
\newcommand{\Rsnark}{\R}
\newcommand{\subvec}[1]{\ensuremath{#1|_{\subspace}}\xspace}
\newcommand{\restricttovec}[1]{\ensuremath{#1|_{\subgroup}}\xspace}
\newcommand{\isinvanishing}[1]{\ensuremath{\mathsf{IsInVanishing_{\subspace,#1}}}\xspace}
\newcommand{\batchedisinvanishing}[1]{\ensuremath{\mathsf{BatchedIsInVanishing_{\subspace,#1}}}\xspace}
\newcommand{\isconsistent}{\ensuremath{\mathsf{IsConsistent}}\xspace}
\newcommand{\isintable}{\ensuremath{\mathsf{IsInTable}}\xspace}
\newcommand{\lincheck}{\ensuremath{\mathsf {lincheck}}\xspace}
\newcommand{\accprot}{\ensuremath{\mathsf {accProt}}\xspace}
\newcommand{\isinvanishingtable}[1]{\ensuremath{\mathsf{IsInVanishingTable_{\subspace,#1}}}\xspace}
\newcommand{\isvanishingsubtable}[1]{\ensuremath{\mathsf{IsVanishingSubtable_{#1}}}\xspace}
\newcommand{\haslowerdegree}{\ensuremath{\mathsf{HasLowerDegree}}\xspace}
\newcommand{\haslowdegree}[1]{\ensuremath{\mathsf{HasLowDegree_{#1}}}\xspace}
\newcommand{\issubtable}[1]{\ensuremath{\mathsf{IsSubtable_{#1}}}\xspace}
\newcommand{\isinsubtable}[2]{\ensuremath{\mathsf{IsInSubtable_{#1,#2}}}\xspace}
\newcommand{\secbasezero}[1]{\ensuremath{\hat{\tau}_{#1}}\xspace}
\newcommand{\secbase}[1]{\ensuremath{\hat{\tau}_{#1}}\xspace}
\newcommand{\secbasereg}[1]{\ensuremath{\tau_{#1}}\xspace}
\newcommand{\secset}{\ensuremath{I}\xspace}
\newcommand{\pubbase}[1]{\ensuremath{\mu_{#1}}\xspace}
\newcommand{\subspace}{\ensuremath{\mathbb{H}}\xspace}
\newcommand{\subgroup}{\ensuremath{\mathbb{H}}\xspace}
\newcommand{\biggroup}{\ensuremath{\mathbb{V}}\xspace}
\newcommand{\subtable}{\ensuremath{T_0}\xspace}
\newcommand{\tablelang}{\ensuremath{\lang_T}\xspace}
\newcommand{\vanishingtablelang}{\ensuremath{\lang_{\subspace}}\xspace}
\newcommand{\nonorm}[1]{\ensuremath{\Gamma^T_{#1}}\xspace}
\newcommand{\unnorm}[2]{\ensuremath{\Gamma^{#1}_{#2}}\xspace}
\newcommand{\bigspacebase}{\ensuremath{\lambda}\xspace}
\newcommand{\bigspacegen}{\ensuremath{\mathsf{h}}\xspace}
\newcommand{\modvan}[1]{\ensuremath{\mathrm{mod\;}Z_{#1}}\xspace}
\newcommand{\extractevaltable}{\ensuremath{\mathsf{ExtractEvalTable}_{C,\tablegroup}}\xspace}
\newcommand{\witsize}{\ensuremath{n}\xspace}
\newcommand{\witruntime}{\ensuremath{\witsize\log\witsize}\xspace}
\newcommand{\tabsize}{\ensuremath{N}\xspace}
\newcommand{\tabruntime}{\ensuremath{\tabsize\log\tabsize}\xspace}
\newcommand{\tab}{\ensuremath{\mathfrak{t}}\xspace}
 \renewcommand{\a}{\ensuremath{\mathsf{a}}\xspace}
\renewcommand{\b}{\ensuremath{\mathsf{b}_0}\xspace}
\renewcommand{\c}{\ensuremath{\mathsf{c}}\xspace}
\newcommand{\vc}{\ensuremath{\mathsf{c_v}}\xspace}
\renewcommand{\v}{\ensuremath{\mathsf{v}}\xspace}
\renewcommand{\r}{\ensuremath{\mathsf{r}}\xspace}
\newcommand{\f}{\ensuremath{\mathsf{f}}\xspace}
\renewcommand{\g}{\ensuremath{\mathsf{g}}\xspace}
\newcommand{\matrices}{\ensuremath{\F^{n\times n}}\xspace}
\newcommand{\ftwo}{\ensuremath{\mathsf{f}_2}\xspace}
\renewcommand{\p}{\ensuremath{\mathsf{p}}\xspace}
\newcommand{\q}{\ensuremath{\mathsf{q}}\xspace}
\newcommand{\qone}{\ensuremath{\mathsf{q_1}}\xspace}
\newcommand{\s}{\ensuremath{\mathsf{s}}\xspace}
\newcommand{\qa}{\ensuremath{\mathsf{q_a}}\xspace}
\newcommand{\qb}{\ensuremath{\mathsf{q_b}}\xspace}
\newcommand{\m}{\ensuremath{\mathsf{m}}\xspace}
\newcommand{\agam}{\ensuremath{a_\gamma}\xspace}
\newcommand{\gamproof}{\ensuremath{\mathsf{\pi_\gamma}}\xspace}
\newcommand{\zerproof}{\ensuremath{\mathsf{\a}_0}\xspace}
\newcommand{\bgam}{\ensuremath{b_\gamma}\xspace}
\newcommand{\bzergam}{\ensuremath{b_{0,\gamma}}\xspace}
\newcommand{\qbgam}{\ensuremath{Q_{b,\gamma}}\xspace}
\newcommand{\zgam}{\ensuremath{Z_{\bigspace,\gamma}}\xspace}
\newcommand{\betaa}{\ensuremath{\mathbf{\boldsymbol{\beta}}}\xspace}
\newcommand{\deltaa}{\ensuremath{\mathbf{\boldsymbol{\delta}}}\xspace}
\newcommand{\gammaa}{\ensuremath{\mathbf{\boldsymbol{\gamma}}}\xspace}
\newcommand{\witcom}{\ensuremath{\mathsf{W}}\xspace}
\newcommand{\instt}{\ensuremath{\Phi^*}\xspace}
\newcommand{\insttbase}{\ensuremath{\Phi}\xspace}
\newcommand{\pow}{\ensuremath{\mathsf{pow}}\xspace}
\newcommand{\eq}{\ensuremath{\mathsf{eq}}\xspace}
\newcommand{\FFT}{\ensuremath{\mathsf{FFT}}\xspace}
\newcommand{\fgam}{\ensuremath{f_{\gamma}}\xspace}
\newcommand{\pgam}{\ensuremath{P_{\gamma}}\xspace}
\newcommand{\supp}[1]{\ensuremath{\mathrm{supp}(#1)}\xspace}
\newcommand{\degoffset}{\ensuremath{\tabsize-1-(\witsize-2)}\xspace}
\newcommand{\modpoly}[1]{\ensuremath{\;\;\mod #1(X) }\xspace}
\newcommand{\coeff}[2]{\ensuremath{(#1)_{[#2]} }\xspace}
\newcommand{\kzg}[1]{\ensuremath{\mathsf{KZG}_{#1,\subgroup}}\xspace}
\newcommand{\accscheme}[2]{$(#1\mapsto #2)$-folding scheme\xspace}
\newcommand{\accrel}{\ensuremath{\rel_{\mathpgoth{acc}}}\xspace}
\newcommand{\relpair}{\ensuremath{\mathfrak{p}}\xspace}
\newcommand{\inststar}{\ensuremath{\inst^*}\xspace}
\newcommand{\witstar}{\ensuremath{\wit^*}\xspace}
\newcommand{\relpairstar}{\ensuremath{\relpair^*}\xspace}
\newcommand{\witt}{\ensuremath{\mathrm{w}}\xspace}
\newcommand{\nodelabel}{\ensuremath{\mathfrak{n}}\xspace}
\newcommand{\roundnum}{\ensuremath{\mathpgoth{k}}\xspace}
\newcommand{\manyvar}{\ensuremath{\mathfrak{J}}\xspace}
\newcommand{\nmin}{\ensuremath{[n]_0}\xspace}
\newcommand{\zfin}{\ensuremath{z_{\mathscr{final}}}\xspace}
\newcommand{\relapp}{\ensuremath{\rel_{app}}\xspace}
\newcommand{\relexec}{\ensuremath{\rel_{\mathrm{exec}}}\xspace}
\newcommand{\relF}{\ensuremath{\rel_F}\xspace}
\newcommand{\init}{\ensuremath{\mathsf{init}}\xspace}
\newcommand{\add}{\ensuremath{\mathsf{add}}\xspace}
\newcommand{\del}{\ensuremath{\mathsf{del}}\xspace}
\renewcommand{\read}{\ensuremath{\mathsf{read}}\xspace}
\newcommand{\countrange}{\ensuremath{[M]}\xspace}
\newcommand{\true}{\ensuremath{\mathsf{true}}\xspace}
\newcommand{\false}{\ensuremath{\mathsf{false}}\xspace}
\newcommand{\finstate}{\ensuremath{V}\xspace}
\newcommand{\ops}{\ensuremath{\mathcal{O}}\xspace}
\newcommand{\op}{\ensuremath{\mathscr{op}}\xspace}
\newcommand{\instapp}{\ensuremath{\mathfrak{x}}\xspace}
\newcommand{\witapp}{\ensuremath{\mathfrak{w}}\xspace}
\newcommand{\instnoops}{\ensuremath{\mathbf{x}}\xspace}
 \renewcommand{\path}{\ensuremath{\mathbf{p}}\xspace}
\renewcommand{\root}{\ensuremath{\mathbf{r}}\xspace}
\newcommand{\predinst}{\ensuremath{\mathpgoth{f}}\xspace}
\renewcommand{\empty}{\ensuremath{g_{\mathscr{empty}}}\xspace}
\newcommand{\funcs}{\ensuremath{\mathrm{F}}\xspace}
\newcommand{\ztafuncs}{\ensuremath{\mathcal{D}}\xspace}
% \newcommand{\instF}{\ensuremath{\mathrm{x}}\xspace}
% \newcommand{\witF}{\ensuremath{\mathrm{w}}\xspace}
\newcommand{\instexec}{\ensuremath{\mathrm{x_{exec}}}\xspace}
\newcommand{\witexec}{\ensuremath{\mathrm{w_{exec}}}\xspace}
\newcommand{\witf}{\ensuremath{\mathsf{w_f}}\xspace}
\newcommand{\sel}{\ensuremath{\mathsf{q}}\xspace}
% \newcommand{\perm}{\ensuremath{\mathsf{s}}\xspace}
\newcommand{\args}{\ensuremath{\mathsf{args}}\xspace}
\newcommand{\callnum}{\ensuremath{\mathsf{c}}\xspace}
\newcommand{\recset}{\ensuremath{\mathsf{V}}\xspace}
\newcommand{\tree}{\ensuremath{\mathsf{T}}\xspace}
\newcommand{\node}{\ensuremath{\mathsf{n}}\xspace}
\newcommand{\incsum}{\ensuremath{\text{\calligra{s}}}\xspace}
\newcommand{\inchash}{\ensuremath{\mathscr{h}}\xspace}
\newcommand{\shlomomath}[1]{\ensuremath{\text{\usefont{U}{BOONDOX-cal}{m}{n}#1}}\xspace}
\newcommand{\calligmath}[1]{\ensuremath{\text{\calligra{#1}}}\xspace}
\newcommand{\ext}{\ensuremath{\mathbf{E}}\xspace}
\newcommand{\finpred}{\shlomomath{f}}
\newcommand{\zksnark}{zk-SNARK\;}
\newcommand{\n}{\shlomomath{n}}
\newcommand{\dimK}{\calligmath{k}}
\newcommand{\M}{\calligmath{M}}
\begin{document}
    \maketitle
\begin{abstract}
The goal of this note is to describe and analyze a simplified variant of the zk-SNARK construction used in the Aztec protocol.
Taking inspiration from the popular notion of Incrementally Verifiable Computation\cite{ivc} (IVC)
we define a related notion of \emph{Repeated Computation with Global state} (RCG). As opposed to IVC, in RCG we assume the computation terminates before proving starts, and in addition to the local transitions some global consistency checks of the whole computation are allowed. However, we require the space efficiency of the prover to be close to that of an IVC prover not required to prove this global consistency.
We show how RCG is useful for designing a proof system for a private smart contract system like Aztec.
\end{abstract}
\section{Introduction}
Incrementally Verifiable Computation (IVC) \cite{ivc} and its generalization to Proof Carrying Data (PCD) \cite{pcd} are useful tools for constructing space-efficient SNARK provers\cite{BCCT}.
In IVC and PCD we always have an acyclic computation.
However code written in almost any programming language \emph{is} cyclic in the sense of often relying on internal calls - 
we start from a function $A$, execute some commands, go into a function $B$, execute its commands, and go back to $A$.
When making a SNARK proof of such an execution, we typically linearize or ``flatten'' the cycle stemming from  the internal call, in one of the following two ways.
\begin{enumerate}
 \item The monolithic circuit approach - we ``inline'' all internal calls (as well as loops) into one long program without jumps.
 \item The VM approach - assume the code of $A,B$ is written in some prespecified instruction set. The program is executed by initially writing the code of $A,B$ into memory, and loading from memory and executing at each step the appropriate instruction according to a program counter. For example, the call to $B$ is made by changing the counter to that of the  first instruction of $B$. To prove correctness of the execution, all we need is a SNARK for proving correctness of a certain number of steps of a machine with this instruction set, and some initial memory state.
\end{enumerate}

The second approach is more generic, while the first offers more room for optimization, so we'd want to use it in resource-constrained settings, e.g. client-side proving.


However, what if we're in a situation where $A$ and $B$ have already been ``SNARKified'' separately?
Namely, there is a verification key attached to each one, and we are expected to use these keys specifically.
This is what happens in the Aztec system.
\paragraph{The Aztec private contract system:}
 Similar to Ethereum - we have contracts; and the contracts have functions.
A function in a contract can internally call a different function in the same or a different contract. Moreover, while writing the code for the different functions,
we can't predict specifically what function will be internally called by a given contract function. For example, a ``send token'' function could have an internal call to an ``authorize'' function.
But the call to ``authorize'' need not be tied to a specific implementation, and consequently verification key - as different token holders are allowed to set their own ``authorize'' function.


The goal of the Aztec system is to enable constructing zero-knowledge proofs of such contract function executions.
For this purpose, a contract is deployed by 
\begin{enumerate}
\item  Computing a verification key for each function of the contract.
\item Adding a commitment to the verification keys of the contract in a global ``function tree''. More accurately, a leaf of this tree is a hash of the
contract address with a Merkle root of a tree whose leaves are the verification keys of that contract's functions.

\end{enumerate}

\paragraph{Dealing with Global state}
Each contract has a set of notes - which are simply  values in a field \F. 
While running, a function can read, add or delete notes belonging to its contract.
We can thus think of the notes as global variables shared between the different functions.

Assume all functions in this system return \acc or \rej. (We can always move the output into the arguments if a function is not of this form.)
Here's a natural way to prove the mentioned execution: Put the arguments to $B$ in the public inputs of both the circuits of $A$ and $B$.
Verify the proofs $\prf_A,\prf_B$ for $A,B$; and check via the public inputs the same value was used in both proofs for the arguments of $B$.

This however, doesn't yet deal with the notes. During execution, note operations happened in a certain order.
We can thus assign a \emph{counter} equal to one to the first operation, and increment the counter with each operation.
We then need to check, for example, that if a note was read with a certain counter, it was indeed added
with a smaller counter.
We can include a description of the note operations performed by a function in the inputs of its circuit. This description will contain the operation type - \set{\add,\read,\del}, the note value, and the counter.
The issue is, what if $A$ is reading a note that was added in the internal call to $B$?
Checking the existence of an $\add$ operation with smaller counter requires a constraint \emph{between} the inputs of both circuits. And for an execution consisting of more calls, this constraint can involve any two circuits in the call tree. 
% Describing the issue more generally, since we are forced to prove things in a different order than they were executed, we must enforce a global consistency between the witnesses of all iterations.

This brings us to the notion of \emph{Repeated Computation with Global state} (RCG). In RCG we have a transition predicate taking us from one state to the next. We wish to prove we know a sequence of witnesses taking us from a legal initial state to a certain publicly known final state. This might remind the reader of the popular notion of \emph{incrementally verifiable computation} (IVC). There are two differences.

\begin{itemize}
 \item 
In RCG we are not interested in ``incremental'' proofs of one step, only in proofs for a whole
sequence of transitions ending in a desired final state.

\item In RCG we also have a \emph{final predicate} checking a joint consistency condition between witnesses from all iterations.

\end{itemize}
 One could ask, why not \emph{only} have a final predicate that includes the transition checks? In other words, a monolithic circuit for the whole computation.
The point is that in our use case the final predicate is applied to small parts of each iteration's witness - namely the note operations. As a result, the decomposition into a transition and final predicate can facilitate obtaining better prover efficiency, especially in terms of prover space.
Roughly, we'll require space sufficient for storing the inputs to the final predicate, in addition to the space required to prove a single transition.

\subsection{Related work}
Recent work  \cite{moonmoon, mangrove} as well as ongoing work \cite{jenstalk} uses folding schemes\cite{nova} to break up proving statements about large computation into smaller statements.
The objective being reducing prover memory and/or improving prover parallelism.
These works, as far as we know, have not formally defined a notion like RCG, and rather use the IVC terminology. However we believe they implicitly use RCG, because of the need to check global constraints between the different chunks.



\section{Preliminaries}
\subsection{Terminology and Conventions}\label{sec:terminology}
We assume all algorithms described receive as an implicit parameter the security parameter $\lambda$.
Similarly, all integer parameters in the paper are implicitly functions of $\lambda$, and of size at most \poly unless explicitly stated otherwise.

Whenever we use the term \emph{efficient}, we mean an algorithm running in time \poly. Furthermore,
we assume an \emph{object generator} \obgen that is run with input $\lambda$ before all protocols, and returns all fields and groups used. Specifically, in our protocol $\obgen(\lambda) = (\F, \G,g)$ where
\begin{itemize}
\item \F is a field of \textbf{prime} size $r = \lambda^{\omega(1)}$
.
\item $\G$ is a group of size $r$.
\item $g$ is a uniformly chosen generator of \G.
\end{itemize}
We usually let the $\lambda$ parameter be implicit, e.g. write \F instead of $\F(\lambda)$.
We denote by \polysofdeg{d} the set of univariate polynomials over \F of degree smaller than $d$. 
We write \G additively.
% We use the notations $\enc1{x}\defeq x\cdot g_1$ and $\enc2{x}\defeq x\cdot g_2$.

We often denote by $[n]$ the integers \set{1,\ldots,n}.
% For example, when we refer below to the field $\F$, it is in fact a function $\F(\lambda)$ of $\lambda$, and part of
% the output of $\obgen(\lambda)$.
We use the acronym e.w.p. for ``except with probability''; i.e. e.w.p. $\gamma$ means with probability \emph{at least} $1-\gamma$.


\paragraph{Representing \G}
Assume an injective function $R:\G \to \F^2$.
Whenever we discuss $a\in \G$ we assume it is represented as  $R(a)$.
When we say for $b\in \F^2$ that $b\in \G$ we mean that there exists $a\in \G$ with $R(a)=b$.

\subsection{Zero-Testing Assumption}
Throughout this paper, we'll make use of a variant of the \emph{Zero-Testing Assumption (ZTA)} from \cite{novarecursive} given in Definition \ref{dfn:ZTA} as well as its generalization, the \emph{$t$-variate Zero-Testing Assumption}, we introduce in Definition \ref{dfn:multiZTA}.

\begin{dfn}\label{dfn:ZTA}
Fix $\cm:\F^M\to K$, hash function $\hash:\B^*\to \F$, and integer $d$. Fix the family of functions \ztafuncs. 
We say the tuple $(D,x,\tau)$ is a \emph{degree $d$-relation for $(\ztafuncs,\hash,\cm)$} if
\begin{enumerate}
 \item $D\in \ztafuncs$.
 \item $f(X)\defeq D(x,\tau)$ is a non-zero element of \polysofdegeq{d}.
 \item Setting $z\defeq \hash(\cm(x),\tau)$, we have $f(z)=0$.
\end{enumerate}
The Zero-Testing Assumption (ZTA) for $(\ztafuncs,\hash,\cm,d)$ states that for any efficient \adv, the probability that 
\adv outputs a degree $d$-relation for $(\ztafuncs,\hash,\cm)$ is \negl.
\end{dfn}

\begin{dfn}[$t$-variate ZTA]\label{dfn:multiZTA}
Fix $\cm:\F^M\to K$, hash function $\hash:\B^*\to \F$, and integer $d$. Fix the family of functions \ztafuncs.
We say the tuple $(D,x,\tau)$ is a \emph{$(t,d)$-relation for $(\ztafuncs,\hash,\cm)$} if
\begin{enumerate}
 \item $D\in \ztafuncs$.
 \item $f(X_1,\ldots,X_t)\defeq D(x,\tau)$ is a non-zero element of $\F_{\leq d}[X_1,\ldots,X_t]$.
 \item Setting $z_i\defeq \hash(\cm(x),\tau,i)$ for $i\in[t]$, we have $f(z_1,\ldots,z_t)=0$.
\end{enumerate}
The $t$-variate Zero-Testing Assumption (ZTA) for $(\ztafuncs,\hash,\cm,d)$ states that for any efficient \adv, the probability that
\adv outputs a $(t,d)$-relation for $(\ztafuncs,\hash,\cm)$ is \negl.
\end{dfn}

We prove that the univariate ZTA implies the $t$-variate ZTA.

\begin{lemma}
Fix a family of functions \ztafuncs whose outputs are polynomials in $\F_{\leq d}[X_1,\ldots,X_t]$. Let $\ztafuncs_t$ be a family of functions to be defined in the proof. Then the univariate ZTA for $(\ztafuncs_t,\hash,\cm,d)$ implies the $t$-variate ZTA for $(\ztafuncs,\hash,\cm,d)$ and $t=\poly$.
\end{lemma}
\begin{proof}
Let \adv be an adversary against the $t$-variate ZTA that outputs the $(t,d)$-relation $(D,x,\tau)$ for $(\ztafuncs,\hash,\cm)$. We construct the adversary \advprime against the univariate ZTA that outputs a degree-$d$ relation for $(\ztafuncs_t,\hash,\cm)$, where $\ztafuncs_t$ will be the union of all the functions $D_i$ defined in the following.

Write $f(X_1,\ldots,X_t)\defeq D(x,\tau)$ as a polynomial in $X_t$ over $\F[X_1,\ldots,X_{t-1}]$:
\[f(X_t)=\sum_{i=0}^d C_i(X_1,\ldots,X_{t-1}) X_t^i.\]
For $i\in[t]$, denote $z_i\defeq\hash(\cm(x),\tau,i)$. Suppose first that $f(z_1,\ldots,z_{t-1},X_t)\not\equiv 0$. Then \advprime can output the degree-$d$ relation $(D_t,x,\tau_t)$, where $D_t$ is the function that computes $f_t(X)\defeq f(z_1,\ldots,z_{t-1},X)$ given $x$ and $\tau_t\defeq(\tau,t)$, deriving $z_1,\ldots,z_{t-1}$ via \hash as part of its operation. Note that $f_t(z_t)=0$ with $z_t=\hash(\cm(x),\tau_t)$.

Otherwise, there is a non-zero polynomial $C_i\in\F_{\leq d}[X_1,\ldots,X_{t-1}]$ which satisfies $C_i(z_1,\ldots,z_{t-1})=0$. If $C_i(z_1,\ldots,z_{t-2},X_{t-1})\not\equiv 0$, \advprime can output the degree-$d$ relation $(D_{t-1},x,\tau_{t-1})$, where $D_{t-1}$ is the function that computes $f_{t-1}(X)\defeq C_i(z_1,\ldots,z_{t-2},X)$ given $x$ and $\tau_{t-1}\defeq(\tau,t-1)$.

Recursively define degree-$d$ relations $(D_i,x,\tau_i)$ until $i=1$ and we have a univariate non-zero polynomial $C_j'\in\F_{\leq d}[X_1]$ with $C_j'(z_1)=0$. In this base case, \advprime can output the degree-$d$ relation $(D_1,x,\tau_1)$, where $D_1$ is the function that computes $f_1(X)\defeq C_j'(X)$ given $x$ and $\tau_1\defeq(\tau,1)$, finishing the proof.
\end{proof}

\section{The execution model:}
We present a formal framework that will be convenient for our proof system of what it means to prove an execution where functions can call each other, and there is global state.
For this, we first introduce record operations which is our specific notion of operating on a global state. 
\subsection{Record operations}
\emph{Records} are pairs $(\v,\c)$ - where $\v\in \F$ is the \emph{value}, and $\c\in \countrange$ is a \emph{counter}.
A \emph{record operation} has one of the following forms:
\begin{itemize}
 \item $(\add,\v,\c)$,
\item $(\del,\v,\vc,\c)$,
\item $(\read,\v,\vc,\c)$.

\end{itemize}
Here $\v\in \F$ is a value and $\c,\vc \in \countrange$ are counters.
$\c$ is interpreted as the counter of the current operation. $\vc$
  is interpreted as the counter of the operation where
the note was added in the case of a \read or \del operation.



We say a sequence \ops of record operations of size $M$ is \emph{consistent} if
\begin{enumerate}
\item The counter values $\c$ are distinct in all elements of \ops, and as a set equal to \countrange.
\item The $\vc$ fields in all $\del$ operations $(\del,\v,\vc,\c)\in \ops$ are distinct.
\item If $(\del,\v,\vc,\c)\in \ops$, then $\vc<\c$ and $(\add,\v,\vc)\in \ops$.
\item If $(\read,\v,\vc,\c)\in \ops$, then $\vc<\c$ and $(\add,\v,\vc)\in \ops$.
\end{enumerate}

Let \recset be a set of records.
We say $\ops$ is \emph{has output \recset} if:
\begin{itemize}
 \item $\ops$ is consistent.
 \item $\recset=\set{(\v,\c) \mid (\add,\v,\c)\in \ops \land \forall \c'\in \countrange,(\del,\v,\c,\c')\notin \ops }$. In words,
 \recset is the set of notes that were added and not deleted.
\end{itemize}



\subsection{The Plonkish relation}
Now we introduce a relation describing the individual function executions tailored to make it convenient to later discuss an execution of a  sequence functions calling each other.
Some choices of constants - like \args being of size four, are arbitrary.
We require the instance to adhere to a form containing both the record operations and the details of the inner calls (though they will be interpreted as such only in the next section when we discuss valid executions).

We fix a polynomial $G:\F^8\to \F$, and integers $N,\n$ that are implicit parameters in the following definition of the relation \relapp.

\relapp consists of all pairs $(\instapp, \witapp)$ having the form 
\begin{itemize}
 \item 
$\instapp= (\f,\args,\callnum, \f_1,\args_1, \f_2,\args_2,\ops)$
where $\f,\f_1,\f_2 \in \G,\args\in \F^4,\callnum\in \set{0,1,2}$; 
\item $\witapp=(\witf,\wit)$
where 
\begin{itemize}
 \item 
$\witf=(\perm_1,\ldots,\perm_4,\sel_1,\ldots,\sel_4)$,where $\perm_j \in [|\instapp|+N]^{\n},\sel_j \in \F^\n$ for each $j\in [4]$
\item $\wit\in \F^N$
\end{itemize}
\end{itemize}
such that
\begin{enumerate}
                                                                                
\item  Setting $x=(\instapp,\wit)$, for all $i\in [\n]$
\[G(\sel_{1,i},\ldots,\sel_{4,i},x_{\perm_{1,i}},\ldots,x_{\perm_{4,i}})=0.\]
\item $\f=\cm(\witf)$.
\end{enumerate}

\subsection{Valid execution trees}\label{sec:validexec}

By an \emph{execution tree of length $n$} we mean a binary tree \tree with $n$ vertices, whose nodes are
labeled by pairs $(\instapp,\witapp)$.
Let \funcs be a set of elements of \G.
Given such \tree we say it is a \emph{valid execution of length $n$ with function set \funcs and output \recset} if
\begin{enumerate}
 \item For each $\node\in\tree$, its label $(\instapp,\witapp)$ is in \relapp.
    \item For each $\node\in \tree$, let $(\instapp,\witapp)$ be its label. Let 
$\instapp= (\f,\args,\callnum, \f_1,\args_1, \f_2,\args_2,\ops)$. Then
    \begin{itemize}
    \item $\f\in \funcs$.
     \item The number of its children is \callnum.
     \item For $i\in [\callnum]$, let $(\f^i,\args^i,\callnum^i, \f^i_1,\args^i_1, \f^i_2,\args^i_2,\ops^i)$ denote the first component of $\node$'s $i$'th child's label. Then $\f_i=\f^i$ and  $\args_i=\args^i$.
     \item Let \ops be the multi-set union of $\instapp.\ops$ over all nodes' labels $(\instapp,\witapp)$. Then \ops has output \recset.
     
    \end{itemize}

\end{enumerate}
Given a set of group elements \funcs say it has \emph{Merkle root \root} if \root is the root of a Merkle tree with the elements of  \funcs at the leaves using some pre-determined encoding.

We define a relation \relexec capturing knowledge of an execution of bounded length with a certain output set of records.
$\relexec$ consists of the pairs $(\instexec,\witexec)$
of the form 
\begin{itemize}
 \item $\instexec=(\root,C,\recset)$,
 \item $\witexec=(n,\tree)$,
\end{itemize}
such that $n\leq C$, and  \tree is a valid execution tree of length $n$ with function set \funcs having Merkle root \root, and output set \recset. 

\section{Repeated Computation with Global state}
An RCG relation is defined by a pair of functions $(F,\finpred)$.\\
\noindent
We call $F(Z,W,Z^*,S)\to \set{\acc,\rej}$ the \emph{transition predicate}.

We informally think of 
\begin{itemize}
\item $Z$ as the public input and $W$ as the private input of $F$.
\item $Z^*$ as the output of $F$ (although the actual output is \set{\acc,\rej}).
\item $S$ as the part of the private input that will be used in the final predicate.
\end{itemize}

Let $D_1,D_2$ be the domains of $S,\recset$ respectively. \finpred is a function
$\finpred:D_1^*\times D_2 \to \set{\acc,\rej}$ called the \emph{final predicate}.

The relation $\rel=\rel_{F,\finpred}$ consists of pairs $(\inpF,\witF)$ such that
$\inpF=(\zfin,C,\recset),\witF=(n,z=(z_0,\ldots,z_n),w=(w_1\ldots,w_n),s=(s_1,\ldots,s_n))$ such that
\begin{itemize}
 \item $z_0.\init = \true$.
 \item $z_n=\zfin$.
 \item $n\leq C$.
 \item For each $i\in [n]$, $F(z_{i-1},w_i,z_i,s_i)=\acc$.
 \item $\finpred(z_n,s_1,\ldots,s_n,\recset)=\acc$.
\end{itemize}


We say a \zksnark for \rel is \emph{space-efficient} if given $s$ and streaming access
to $z$ and $w$ \prv requires space $O(|F|+|s|+\log n)$.


\subsection{Valid executions as RCGs}\label{sec:exec->RFC}
We show how to represent checking valid executions as defined in Section \ref{sec:validexec} via RCGs.

Define the function $F(Z,W,Z^*,S)\to \set{\acc,\rej}$ as follows.
\begin{itemize}
 \item $Z=(g,\root,\init)$ where $g$ is a stack of elements of the form $(\f,\args)$, \root is a root of a Merkle tree, and \init a boolean.
 \item $Z^*=(g^*,\root^*,\init^*)$ has the same form.
 \item $W=(\path,\instnoops,\witapp)$
 \item $S$ is a set of record operations.
\end{itemize}

\noindent Under this notation
$F(Z,W,Z^*,S)=\acc$ if and only if
\begin{enumerate}
\item If $\init=\true$, $g$ contains exactly one element.
 \item Denoting $g[0]=(\f,\args)$, we have $\f=\instnoops.\f$ and $\args=\instnoops.\args$.
 \item Setting $\instapp=(\instnoops,S)$, we have $(\instapp,\witapp)\in \relapp$.
 \item \path is a Merkle path from \f to \root.
 \item  $\root=\root^*$.
 \item $g^*$ is the result of popping $(\f,\args)$ from $g$ and then pushing the $\instapp.\callnum$ elements
 $(\instapp.\f_i,\args_i)$ for $i\in [\instapp.\callnum]$.
\end{enumerate}


Denote by \empty the empty stack.
We define the function \finpred on input $(z_n,s_1,\ldots,s_n,\recset)$ to output \acc if and only if  $z_n.g=\empty$,
each $s_i$ is a well-formed set of record operations, and when defining \ops as the multi-set union of $s_1,\ldots,s_n$ it  has output \recset.




\begin{lemma}\label{lem:execasRCG}
There is an efficiently computable and efficiently invertible map $\varphi$ such that the following holds.
Let \funcs be a set of function commitments with Merkle root \root. Fix positive integers $n,C$ with $n\leq C$.
Define $\zfin=(\empty,\root,\false)$. Let \tree be an execution tree of length $n$.

Then  $( (\root,C,\recset),\tree)\in \relexec$ if and only if $((\zfin,C,\recset),\varphi(\tree))\in \rel_{F,\finpred}$.

 
\end{lemma}
\begin{proof}
We describe the operation of $\varphi$.
Given \tree of length $n$ let $(\instapp_1,\witapp_1),\ldots,(\instapp_n,\witapp_n)$ be the labels of its nodes according to
DFS order.
Define a sequence of stacks $g_0,\ldots,g_n$ according to the sequence of labels.

Namely, $g_0$ is the stack containing only $(\instapp_1.\f,\instapp_1.\args)$. And for each $i\in [n]$, $g_i$ is
the stack obtained by popping $g[0]$ and adding $(\instapp_i.\f_j,\instapp_i.\args_j)$ for $j\in [\instapp_i.\callnum]$.

Now, define $z_0=(g_0,\root,\true)$ and for each $i\in [n]$, $z_i=(g_i,\root,\false)$.



We now need to refer to the record operations in each instance separately.
For this purpose,
for each $i\in [n]$, denote $\instapp_i=(\instnoops_i,\ops_i)$.
For each $i\in [n]$, let $\path_i$ be the path from $\instapp_i.\f$ to \root.
Define for each $i\in [n]$, $w_i=(\path_i,\instnoops_i,\witapp_i)$, $s_i=\ops_i$.
Finally set $z=(z_0,\ldots,z_n),w=(w_1,\ldots,w_n),s=(s_1,\ldots,s_n)$
 and $\varphi(\tree)=(n,z,w,s)$. Given this definition of $\varphi$ the statement of the lemma is straightforward to check.
\end{proof}

\section{Removing the global state}\label{sec:Fstar}

\paragraph{Rational Identities:}
Following work on ``log-derivative lookup'' \cite{bplusplus,logup}, we characterize the validity of record operations via rational identites.
\begin{claim}\label{clm:reducetologder}
Assume $\F$ has characteristic larger than $n+1$.
Let \recset be a set of records and $\ops=\sett{(\op_i,\v_i,\vc_i,\c_i)}{i\in [n]}$ be a set of record operations (defining $\vc_i=0$ when $\op_i=\add$) with $\vc_i<\c_i$ for all $i\in [n]$.
Then \ops has output \recset if and only if the following rational function identities hold:
\begin{enumerate}
 \item \[\sum_{(\v,\c)\in \recset}\frac{1}{X+\v Y+\c}=\sum_{i\in [n],\op_i=\add}\frac{1}{X+\v_i Y+\c_i}-\sum_{i\in [n], \op_i=\del}\frac{1}{X+\v_i Y+\vc_i}.\]
 \item For some $m\in \F^n$,
 we have
 \[\sum_{i\in [n],\op_i=\add}\frac{m_i}{X+\v_i Y+\c_i}=\sum_{i\in [n], \op_i=\read}\frac{1}{X+\v_i Y+\vc_i}.\]
 \item \[\sumi{n}\frac{1}{X+\c_i}=\sumi{n}\frac{1}{X+i}.\]
\end{enumerate}
\end{claim}
\begin{proof}

We focus on the only if direction. That is, if \ops doesn't have output \recset one of the three identities should not hold.
 Let $R_{v,c}\defeq \frac{1}{X+vY+c}$. The main fact we use is that the rational functions $\sett{R_{v,c}}{(v,c)\in \F^2}$ are linearly independent.
 Thus, $\sum_{v,c} a_{v,c} R_{v,c} = \sum b_{v,c} R_{v,c}$ implies $a_{v,c}=b_{v,c}$ for each $(v,c)\in \F^2$.
 The event of \ops not having output \recset means one of the following occurs.
 \begin{enumerate}
  \item  The multi-set of counters $\sett{\c_i}{i\in [n]}$ doesn't equal \set{1,\ldots,n}. In this case, the LHS of the third identity will not have all one coefficients for the elements \sett{R_{0,\c_i}}{i\in [n]} and so cannot equal the RHS.
  
  Note that when we are not in this case the counters in \ops are all distinct, which we assume for the next cases.
 \item For some $\v,\vc,\c$, $(\del,\v,\vc,\c)\in \ops$ but $(\add,\v,0,\vc)\notin \ops$; or for some $\v,\vc,\c_1\neq \c_2$, $(\del,\v,\vc,\c_1),(\del,\v,\vc,\c_2)\in \ops$: In the first identity RHS, we will have $R_{\v,\vc}$ with coefficient in the range  $\set{-1,\ldots,-n}$, while in the LHS it has coefficient one or zero.
 \item For some $\v,\vc,\c$, $(\read,\v,\vc,\c)\in \ops$ but $(\add,\v,0,\vc)\notin \ops$: In the second identity RHS $R_{\v,\vc}$ will have a coefficient in the range $\set{1,\ldots,n}$ while in the LHS it has coefficient zero.
 \item \recset is \emph{not} equal to the set $\recset'$ of $(\v,\c)$ for which $(\add,\v,0,\c)\in \ops$ but $(\del,\v,\c,\c')\notin \ops$. We look at the first identity. $\recset'$ is precisely the set of $(\v,\c)$ with coefficient one on the RHS, while \recset is the set of $(\v,\c)$ with coefficient one on the LHS. Hence the first identity cannot hold in this case.
 \end{enumerate}
\end{proof}







We reduce the relation $\rel_{F,\finpred}$ from the last section to 
$\rel_{F^*,\finpred^*}$:


Let $F$ denote the function $F(Z,W,Z^*,S)\to \set{\acc,\rej}$ as from Section \ref{sec:exec->RFC}.

Define the function $F^*:(Z,W,Z^*)\to \set{\acc,\rej}$:
\begin{itemize}
 \item $Z=(Z_F,\inchash,\incsum,\alpha,\beta,\eps)$.
 \item $Z^*=(Z_F^*,\inchash^*,\incsum^*,\alpha^*,\beta^*,\eps^*)$.
 \item $W=(W_F,S_F,m)$
\end{itemize}
\noindent
Under this notation
$F^*(Z,W,Z^*)=\acc$ if and only if
\begin{enumerate}
\item $F(Z_F,W_F,Z_F^*,S_F)=\acc$.
\item If $Z_F.\init = \true$, $\inchash=\emptyset$ and $\incsum=0$.
\item Let $S_F=\sett{(\op_i,\v_i,\vc_i,\c_i)}{i\in [k]}$. We have $\vc_i<\c_i$ for all $i\in [k]$, and \\
$\incsum^*=\incsum\;+$
\[\sum_{i\in [k];\op_i = \add}\frac{1+ \eps m_i}{\alpha +\beta \v_i+\ \c_i}-\sum_{i\in [k];\op_i = \read}\frac{\eps}{\alpha +\beta \v_i+\vc_i}-\sum_{i\in [k];\op_i = \del}\frac{1}{\alpha +\beta \v_i+ \vc_i}+ \sumi{k}\frac{\eps^2}{\alpha+\c_i}.\]
\item $\alpha^*=\alpha, \beta^*=\beta,\eps^*=\eps$.
\item $\inchash^*=\hash(\inchash,\cm(S_F,m))$. \\
\end{enumerate}

\noindent $\finpred^*(Z,\recset)=\acc$ if and only if
\begin{enumerate}
 \item $\incsum=\sum_{(\v,\c)\in \recset} \frac{1}{\alpha + \beta \v+ \c}+\sum_{i\in \countrange}\frac{\eps^2}{\alpha+i}$,
 \item $\hash(\inchash,\recset,1)=\alpha$, $\hash(\inchash,\recset,2)=\beta$, $\hash(\inchash,\recset,3)=\eps$.
\end{enumerate}

\begin{lemma}\label{lem:FtoFstar}
There is an efficiently computable and efficiently invertible map $\varphi$ such that the following holds.
Let \funcs be a set of function commitments with Merkle root \root. Fix positive integers $n,C$ with $n\leq C$.
Fix some $\alpha,\beta,\eps\in \F$ and set of records \recset.
Define $\zfin=(\empty,\root,\false), \zfin^*\defeq (\zfin,\inchash,\incsum,\alpha,\beta,\eps)$ and set $\inst\defeq (\zfin^*,C,\recset)$.
Suppose efficient \adv outputs \wit such that $(\inst,\wit)\in \rel_{F^*,\finpred^*}$.

Set $d\defeq 3M+|\recset|+1$. Define $\cm'((s_1,m_1),\ldots,(s_n,m_n))\defeq h_n$, where $h_0\defeq\emptyset$ and $h_i\defeq\hash(h_{i-1},\cm(s_i,m_i))$ for $i \in [n]$.
Let \ztafuncs be a function family to be defined in the proof.
Assume the 3-variate ZTA holds for $(\ztafuncs,\hash,\cm',d)$.
Then e.w.p. \negl, $((\zfin,C,\recset),\varphi(\wit))\in \rel_{F,\finpred}$.
\end{lemma}
\begin{proof}
Given a witness $\wit=(n,z',w')$ with $z'=(z_0',\ldots,z_n'),w'=(w_1',\ldots,w_n')$ output by \adv, denote
$z_i'=(z_i,\inchash_i,\incsum_i,\alpha_i,\beta_i,\eps_i), w_i'=(w_i,s_i,m_i)$.
Define $\witF=\varphi(\wit)$ as $\witF\defeq (n,(z_0,\ldots,z_n),(w_1,\ldots,w_n),(s_1,\ldots,s_n))$.
From $(\inst,\wit)\in \rel_{F^*,\finpred^*}$,  we know \witF satisfies the transition constraints, namely for $i\in [n]$,
$F(z_{i-1},w_i,z_i,s_i)=\acc$. We also know that $z_0.\init = \true$, $z_n=\zfin$, and $n\leq C$.
It is left to show that e.w.p. \negl $\finpred(z_n,s_1,\ldots,s_n,\recset)=\acc$.

From $((\zfin^*,C,\recset),\wit)\in \rel_{F^*,\finpred^*}$ we know the equations from Claim \ref{clm:reducetologder} hold
at $\alpha,\beta,\eps$. That is, defining the rational function
\begin{align*}
r(X,Y,Z)\defeq&\sum_{(\v,\c)\in \recset}\frac{1}{X+\v Y+\c}-\sum_{i\in \countrange,\op_i=\add}\frac{1}{X+\v_i Y+\c_i}+\sum_{i\in \countrange, \op_i=\del}\frac{1}{X+\v_i Y+\vc_i} \\
&+Z\left(\sum_{i\in \countrange,\op_i=\add}\frac{(m_1\|\ldots\|m_n)_i}{X+\v_i Y+\c_i}-\sum_{i\in \countrange, \op_i=\read}\frac{1}{X+\v_i Y+\vc_i}\right) \\
&+Z^2\left(\sum_{i\in \countrange}\frac{1}{X+\c_i}-\sum_{i\in \countrange}\frac{1}{X+i}\right),
\end{align*}
we have $r(\alpha,\beta,\eps)=0$. If any of the three rational identities from Claim \ref{clm:reducetologder} does not hold, we have $r(X,Y,Z)\not\equiv 0$.
Let $f\in\F_{\leq d}[X,Y,Z]$ denote the resulting non-zero degree-$d$ polynomial when multiplying $r(X,Y,Z)$ with all of its denominators. Note that $f(\alpha,\beta,\eps)=0$. Define $D$ as the function that computes $f(X,Y,Z)$ given $x\defeq((s_1,m_1),\ldots,(s_n,m_n))$ and $\tau\defeq\recset$. Set $\ztafuncs\defeq\{D\}$. Then we can define an efficient adversary \advprime against the $3$-variate ZTA for $(\ztafuncs,\hash,\cm',d)$ that outputs the $(3,d)$-relation $(D,x,\tau)$, which has probability \negl.
\end{proof}





\section{Non-interactive folding schemes}\label{sec:folding}
We fix a vector space $K$ over \F, and an $\F$-linear function $\cm:\F^M\to K$ that will be an implicit parameter in Definition \ref{dfn:accschme}.
We say a relation \rel is \emph{\cm-compatible} if every element of \rel has the form $(\cm(\wit),\wit)$.
We say \rel is \emph{\cm-extendable} if every element of \rel has the form $((\cm(\wit),\tau),\wit)$ for some $\wit,\tau$.
\begin{dfn}\label{dfn:accschme}
Fix relations \rel and \accrel that are respectively \cm-compatible and \cm-extendable. An \emph{\accscheme{\rel}{\accrel}} is a pair of algorithms $(\prv, \ver)$
such that

\begin{enumerate}
 \item \prv on input $(\insttbase,\inst';\wit,\wit')$ produces a pair $(\instt,\wit^*)$ and proof \prf .
 \item \ver on input $(\insttbase,\inst',\instt,\prf)$ outputs \acc or \rej such that
%  \item \ver receives as $\inst,\inst'$ as inputs and outputs \acc or \rej at the end.
\begin{enumerate}
 \item \textbf{Completeness:} If $(\insttbase,\wit)\in \accrel, (\inst',\wit')\in \rel$ and $\prv(\insttbase,\inst';\wit,\wit')=(\prf,\instt,\witstar)$, then with probability $1-\negl$, $(\instt,\witstar)\in \accrel$ and $\ver(\insttbase,\inst',\instt,\prf)=\acc$.
\item \textbf{Knowledge soundness  given extractable commitments:}

For any efficient \adv the probability of the following event is \negl:
\adv outputs $(\inst,\tau),\inst',(\inststar,\tau^*),\wit,\wit',\witstar,\prf$
such that 
\begin{enumerate}
\item $\cm(\wit)=\inst,\cm(\wit')=\inst',\cm(\witstar)=\inststar$, 
\item $\ver(\inst,\inst',\inststar,\prf)=\acc$,
\item $((\inststar,\tau^*),\witstar)\in\accrel$, 
\item $((\inst,\tau),\wit)\notin\accrel$ or $(\inst',\wit')\notin \rel$.
\end{enumerate}

\end{enumerate}
\end{enumerate}
\end{dfn}
\begin{remark}\label{rem:agm->ext commitments}
 The justification for requiring only knowledge soundness given extractable commitments is as follows.
 We assume the Algebraic Group Model \cite{agm} and use a commitment scheme based on linear combination of group elements like \cite{kate}.
 In this model with such a commitment scheme an adversary \adv must output \wit, with $\cm(\wit)=\inst$ whenever it outputs some $\inst\in K$.
 For more details see \cite{agm} or Section 2.2 of \cite{plonk}, as well as Section \ref{sec:recursive} of this paper.
\end{remark}

\subsection{Relations for folding schemes}
We define a more general satisfiability relation than in \cite{protogal}.
We have as parameters, integers $n,\M,d$ and an $\F$-vector space $K$.
We have a
\begin{itemize}
 \item \emph{Constraint function} $f:\F^\M\to \F^n$ which is a vector of $n$ $\M$-variate polynomials of degree $\leq d$,
\item \emph{Instance predicate} $\predinst:K \to \set{\acc,\rej}$,
\item \emph{Commitment function} $\cm:\F^\M \to K$ which is $\F$-linear and assumed
to be collision resistant.
\end{itemize}

Given $(f,\predinst,\cm)$ we define a relation $\rel_{f,\predinst,\cm}$ consisting of all pairs
$(\inst,\wit)$ such that
\begin{enumerate}
 \item $f(\wit)=0^n$.
 \item $\predinst(\inst)=\acc$.
 \item $\inst=\cm(\wit)$.
\end{enumerate}

\paragraph{The relation $\relrand$:}
 For brevity, let $\rel=\rel_{f,\predinst,\cm}$.
  As in \cite{protogal}, we define the ``randomized relaxed'' version of $\rel, \relrand$. First, some required notation.
 Let $t\defeq \log n$. For $i\in [n]$, let $S\subset \set{0,\ldots,t-1}$ be the set such that $i-1=\sumj{S}2^j$. We define the $t$-variate polynomial $\pow_i$ as
 \[\pow_i(X_0,\ldots,X_{t-1}) \defeq \prod_{\ell\in S} X_\ell.\]
 Note that if $\betaa=(\beta,\beta^2,\beta^4,\ldots,\beta^{2^{t-1}})$, $\pow_i(\betaa)=\beta^{i-1}$. 
%  This implies the $n$ \emph{univariate} polynomials 
%  $\pow_i(X,X^2,\ldots,X^{2^{t-1}})$ are linearly independent.
  
  Given the above notation, $\relrand$ consists of the pairs $( \insttbase,\wit)$ with $\insttbase=(\inst,\betaa,e)$ such that
 \begin{enumerate}
  \item $\inst=\cm(\wit)$.
  \item $\betaa\in \F^t, e\in \F$ and we have 
  \[\sumi{n}\pow_i(\betaa) f_i(\wit) = e.\]
 \end{enumerate}
(Here, $f_i$ denotes the $i$'th output coordinate of $f$.) 
\subsection{The \protogal scheme}
Deviating from \cite{protogal}, we explicitly present \protogal as a \emph{non-interactive} folding scheme, and for the special case of folding $k=1$ instances.
We define $Z(X)\defeq X\cdot(1-X)$. We assume below \hash is a function mapping arbitrary strings to elements of \Fstar. \\ \\ \noindent
\underline{$\prv_{PG}(\insttbase=(\inst,\betaa,e),\inst_1;\wit,\wit_1)$:}
\begin{enumerate}
% \item \ver checks for each $i\in [k]$ that $\inst_i$ is \hash-consistent, and outputs \rej otherwise.
%  \item \ver sends a challenge $\delta \in \F$.
 \item Compute $\delta=\hash(\insttbase,\inst_1)$. Define $\deltaa\defeq (\delta,\delta^2,\ldots,\delta^{2^{t-1}})\in \F^t$.
 \item\label{step:computeF} Compute the polynomial 
 \[F(X)\defeq \sumi{n}\pow_i(\betaa+X\deltaa) f_i(\wit).\]
 (Note that $F(0)=\sumi{n} \pow_i(\betaa) f_i(\wit) = e$.)
\item\label{step:sendFcoeffs} Denote the non-constant coefficients of $F$ by $a\defeq (F_1,\ldots,F_t)$.
\item Compute $\alpha = \hash(\insttbase,\inst_1,a)$.
% \item\label{step:computeFalpha} Note that $F(\alpha)= e+\sumi{t}F_i \alpha^i$.
\item\label{step:computeBetastar} Compute $\betaa^*\in \F^t$ where $\betaa^*_i\defeq \betaa_i+\alpha\cdot \deltaa_i$.
\item\label{step:computeG} Define the polynomial $G(X)$ as 
\[G(X)\defeq \sumi{n}\pow_i(\betaa^*) f_i(X\cdot \wit + (1-X)\wit_1).\]
\item Compute the polynomial $K(X)$ such that 
\[G(X)= F(\alpha) X + Z(X) K(X).\]
\item\label{step:sendK} Let $b\defeq (K_0,\ldots,K_{d-2})$ be the coefficients of $K(X)$.
\item Compute $\gamma=\hash(\insttbase,\inst_1,a,b)$.
\item\label{step:compute-estar} Compute 
\[e^* \defeq F(\alpha)\gamma + Z(\gamma)K(\gamma).\]
\end{enumerate}
Finally, output
\begin{itemize}
\item the instance
$\instt=(\inststar,\betaa^*,e^*),$
where \[\inst^*\defeq \gamma\cdot \inst + (1-\gamma) \inst_1,\]


\item the witness 
$\witstar\defeq \gamma\cdot \wit + (1-\gamma)\cdot \wit_1,$
\item and the proof $\prf\defeq (a,b)$.\\
\end{itemize}
\noindent
\underline{$\verpg(\insttbase,\inst_1, \instt,\pi=(a,b)):$}\noindent
\begin{enumerate}
\item Check that $\predinst(\inst_1)=\acc$. Output \rej otherwise.
 \item Compute $\delta,\alpha,\betaa^*, \gamma$ as in the prover algorithm given $\insttbase,\inst_1,a,b$.

\item Check that $\instt$ is computed as in the prover algorithm. Output \acc iff this is the case. 
\end{enumerate}



For the knowledge soundness analysis we'll use a variant of the Zero-Testing Assumption from \cite{novarecursive}, see Definition \ref{dfn:ZTA}.




\begin{thm}\label{thm:pgsoundness}
Set $d'\defeq \max \set{n,d}$. Denote $\cm'((\wit,\betaa,e),\wit_1)\defeq ((\cm(\wit),\betaa,e),\cm(\wit_1))$.
Let \ztafuncs be a family of four functions to be defined in the proof.
Assume that \cm is collision resistant and the ZTA holds for $(\ztafuncs,\hash, \cm',d')$.
Then \protogal is a \accscheme{\rel}{\relrand}.
\end{thm}
\begin{proof}
The main thing to prove is knowledge soundness given extractable commitments.
 Fix any efficient \adv. Let $E$ be the event that \adv outputs
$\insttbase=(\inst,\betaa,e),\inst_1,\instt=(\inststar,\betaa^*,e^*),$ $\wit,\wit_1,\witstar,\prf$ such that 
\begin{enumerate}
\item $\cm(\wit)=\inst,\cm(\wit_1)=\inst_1,\cm(\witstar)=\inststar$,
\item $\verpg(\insttbase,\inst_1,\instt,\prf)=\acc$,
\item $(\instt,\witstar)\in\relrand$,
\item $(\insttbase,\wit)\notin\relrand$ or $(\inst_1,\wit_1)\notin \rel$.
\end{enumerate}
According to Definition \ref{dfn:accschme}, knowledge soundness is equivalent to $E$ having 
probability \negl for any such \adv.
We construct an efficient \advprime that runs \adv, and when $E$ occurs outputs either a collision of 
\cm or a degree $d'$-relation for $(\hash,\cm)$. By the theorem assumption this implies $E$ is contained in two events of
probability \negl, and must have probability \negl itself.

Assume we are in event $E$.
% Since $\ver_{PG}(\insttbase,\inst_1,\instt,\prf)=\acc$, we have $\inststar=\gamma \inst +(1-\gamma)\inst_1$.
% And since $(\instt,\witstar)\in \relrand$ we have $\cm(\witstar)=\inststar$.
Using linearity of \cm, when $E$ occurs we have 
\[\cm( \gamma\wit+(1-\gamma)\wit_1)=\gamma \inst +(1-\gamma)\inst_1=\inststar=\cm(\witstar).\]
Thus, if $\witstar \neq \gamma \wit + (1-\gamma)\wit_1$, \advprime can output $(\witstar,\gamma\wit+(1-\gamma)\wit_1)$
as a collision of \cm.
Now assume that $\witstar = \gamma \wit + (1-\gamma)\wit_1$.
Suppose $\prf=(a,b)$, with $a=(a_1,\ldots,a_t),b= (b_0,\ldots,b_{d-2})$.
Define $F_0(X)\defeq e+\sum_{i\in [t]} a_iX^i,K'(X)\defeq \sum_{i=0}^{d-2} b_iX^i$.
Let $\delta,\deltaa,\alpha,\betaa^*, \gamma$ be computed as in the prover description given $a,b$.
Define the polynomials
\begin{align*}
F'(X) &\defeq F_0(X)-\sum_{i\in [n]} \pow_i(\betaa+X\deltaa) f_i(\wit), \\
G'(X) &\defeq F_0(\alpha)X+ Z(X)K'(X)-\sum_{i\in [n]}\pow_i( \betaa^*) f_i(X\wit +(1-X)\wit_1).
\end{align*}

Since $((\inststar,\betaa^*,e^*),\witstar)\in \relrand$ and $\ver_{PG}(\insttbase,\inst',\instt,\prf)=\acc$,
\begin{align*}
G'(\gamma) &= F_0(\alpha)\gamma + Z(\gamma)K'(\gamma)- \sum_{i\in [n]} \pow_i(\betaa^*) f_i(\gamma\wit +(1-\gamma)\wit_1) \\
           &= e^*-\sum_{i\in [n]} \pow_i(\betaa^*) f_i(\witstar) = e^*-e^* = 0.
\end{align*}

Set $x\defeq ((\wit,\betaa,e),\wit_1)$ and $\tau_1\defeq (a,b)$.
If $G'\not\equiv 0$, \advprime outputs the degree $d'$-relation $(D_1,x,\tau_1)$,
where $D_1$ is the function that computes $G'(X)$ given $x,\tau_1$.

Assume now that $G'\equiv 0$.

If $(\inst_1,\wit_1)\notin \rel$, using $Z(0)=0$ we have 
\[G'(0)= -\sum_{i\in [n]}\pow_i( \betaa^*) f_i(\wit_1)=0.\]
Define the polynomial $A(X)\defeq \sumi{n}f_i(\wit_1)\pow_i(\betaa_1+X \delta,\betaa_2 +X \delta^2,\ldots,\betaa_t+X \delta^{2^{t-1}})$.
We have $A(\alpha)=0$. Suppose first that $A(X)\not\equiv 0$. Then setting $D_2$ to be the function that computes $A(X)$ given $x$ and $\tau_2\defeq a$, \advprime can output the degree $n$ relation $(D_2,x,\tau_2)$. Now assume $A(X)\equiv 0$.
Define the polynomial 
\[B(X,Y)\defeq \sumi{n}f_i(\wit_1)\pow_i(\betaa_1+XY,\betaa_2+XY^2,\ldots,\betaa_t+X Y^{2^{t-1}}).\]
We have that $B(X,\delta)\equiv 0$.
Write $B(X,Y)$ as a polynomial in $X$ over $\F[Y]$:
\[B(X)=\sum_{i=0}^t C_i(Y)X^i.\]
Because we're in the case $(\inst_1,\wit_1)\notin \rel$, $B$ is a combination of the $n$ linearly independent polynomials $\sett{\pow_i(\betaa_1+XY,\betaa_2+XY^2,\ldots,\betaa_t+X Y^{2^{t-1}})}{i\in [n]}$ with at least one non-zero coefficient, and so $B(X,Y)\not\equiv 0$. This means one of the polynomials $C_i$ is non-zero, while $C_i(\delta)=0$.
We can use this to let \advprime output the degree $n$ relation $(D_3,x,\tau_3)$, where $D_3$ is the function that computes $C_i$ given $x$ and $\tau_3\defeq\emptyset$.

Now, assume that $(\insttbase,\wit)\notin \relrand$.
As we're still assuming $G'\equiv 0$, we have 
\[G'(1)=F_0(\alpha)-\sumi{n}\pow_i(\betaa^*) f_i(\wit)=0.\]
But we also have $F'(\alpha)=G'(1)$ and so $F'(\alpha)=0$.
On the other hand,
\[F'(0)=e-\sumi{n}\pow_i(\betaa) f_i(\wit)\neq 0.\]
Setting $\tau_4\defeq a$ and $D_4$ to be the function that computes $F'$ given $x,\tau_4$, we
have that $(D_4,x,\tau_4)$ is a degree $\log n$ relation that \advprime can output in this case.
Setting $\ztafuncs=\set{D_1,D_2,D_3,D_4}$ we have proven knowledge soundness under the theorem assumptions.
\end{proof}

% \section{On protogalaxy}
% \subsection{Introspective constraints}
% Constraints $f_i$ on \wit, simply low degree polynomials, but have the ability to refer to
% components of $\cm(m_j)$
% Function $F$ should have ``introspection'' ability to look at commitments.
% 
% commitment function will output two representations of $\cm(w)$ - in \G and in \F and 
% 
% or PI will include \F representation. which will then be part of $w$. \V will check representations match.
\section{Adversaries supporting recursive extraction}\label{sec:recursive}
Following \cite{novarecursive}, we define a model of ``recursive extraction'' for our analysis in Section \ref{sec:FtoFprime}. We assume our commitment function $\cm:\F^\M\to K$ is surjective.
We assume the existence of an efficiently computable injective representation function $R:K\to \F^{\dimK}$. Whenever an adversary \adv outputs $a\in K$ we assume it is represented as $R(a)$.
When analyzing knowledge soundness of our zk-SNARK in the next section, we put the following limitation on \adv. Say \adv outputs a vector 
$v$ over \F. If there is an index $i$ such that $(v_i,\ldots,v_{i+\dimK-1})=R(a)$ for some $a\in K$, then \adv must also output $\omega\in \F^\M$ such that $\cm(\omega)=a$.
We call such \adv a \emph{recursive algebraic adversary.}


This assumption is motivated by a ``recursive'' interpretation of the Algebraic Group Model\cite{agm} as done in \cite{novarecursive}.
For illustration, consider first the case where $\cm:\F^\n\to\G$ is defined as $\cm(\wit)=<\wit,V>$ for a fixed vector $V\in \G^\n$ derived in a setup procedure. I.e. \cm is a Pedersen commitment in this case.
In this case, the AGM forces \adv when outputting $a\in \G$ to also output $\wit$ such that $<\wit,V>=\cm(\wit)=a$. \cite{novarecursive} now raise the idea that if a prespecified subset of indices of \wit is a valid representation of $b\in \G$, it is reasonable to demand of an algebraic \adv to also output $\wit_2$ with $<\wit_2,V>=b$.\\


Our concrete choice for \cm when using \protogal, is of the following form.\footnote{\cm is of this form because it is derived from an interactive protocol where the prover messages are in committed form, but verifier challenges are in the clear. See \cite{protostar,protogalaxy} for more details.} We have a setup procedure outputting a vector of group elements $V\in \G^\n$ is output. 
We have some fixed partition of our input $\wit \in \F^\M$ into continuous segments of size either one or \n. To get $\cm(\wit)$ we operate segment-wise.
If the segment $\wit_i$ is of size one we simply append $\wit_i$ to the output. If a segment $s=(\wit_i,\ldots,\wit_{i+\n-1})$ is of size \n we append $<s,V>$ to the output.
In particular the output $\cm(\wit)$ is a mixture of \F and \G elements, and accordingly the space $K$ is a direct sum of spaces $\mathbb{V}_i\in \set{\F,\G}$. It follows that an algebraic adversary outputting $c\in K$ must output \wit with $\cm(\wit)=c$ as the AGM forces it to send an opening $s\in \F^\n$ to each $a\in \G$. 



\newcommand{\instrcg}{\ensuremath{\mathscr{x}}\xspace}
\newcommand{\witrcg}{\ensuremath{\mathscr{w}}\xspace}
\newcommand{\empt}{\calligmath{empty}}

 \renewcommand{\acc}{\ensuremath{\mathsf{acc}}\xspace}
 \newcommand{\accprev}{\ensuremath{\mathsf{acc}_{0}}\xspace}%\calligmath{prev}}}\xspace}
 \renewcommand{\inst}{\ensuremath{\mathsf{inst}}\xspace}
 \newcommand{\witacc}{\ensuremath{\mathbf{w\mhyphen}\mathsf{acc}}\xspace}
 \newcommand{\witinst}{\ensuremath{\boldsymbol{w}\mhyphen \mathsf{inst}}\xspace}
 \newcommand{\relfin}{\ensuremath{\rel_{\mathsf{fin}}}\xspace}
 \newcommand{\relrcg}{\ensuremath{\rel_{F,\finpred}}\xspace}
 \newcommand{\instfin}{\ensuremath{\shlomomath{x}}}%_{\mathsf{fin}}}\xspace}
 \newcommand{\witfin}{\ensuremath{\shlomomath{w}}}%_{\mathsf{fin}}}\xspace}
 \newcommand{\emptyacc}{\ensuremath{\acc_{\text{\calligra{empty}}}}\xspace}
%  \newcommand{\verpg}{\ensuremath{\ver_{\cm,n}}\xspace}
 \newcommand{\prvpg}{\ensuremath{\prv^{\mathrm{PG}}_{\cm,n,f}}\xspace}
 \newcommand{\prvsnark}{\ensuremath{\prv_{\mathsf{fin}}}\xspace} %{\calligra{fin}}}}\xspace}
 \newcommand{\versnark}{\ensuremath{\ver_{\mathsf{fin}}}\xspace}%text{\calligra{fin}}}}\xspace}
%  \newcommand{\prvsnark}{\ensuremath{\prv_{\text{\calligra{fin}}}}\xspace}
%  \newcommand{\versnark}{\ensuremath{\ver_{\text{\calligra{fin}}}}\xspace}
 \newcommand{\prffin}{\ensuremath{\pi_{\mathsf{fin}}}\xspace}
\section{The main construction}\label{sec:FtoFprime}
We say  an RCG relation $\rel_{F,\finpred}$ is \emph{trivial} if the $S$ variable is not used in $F$, and accordingly \finpred depends only on $(\zfin,\recset)$.
We assume in this section $R_{F,\finpred}$ is trivial, thinking of it as  $\rel_{F^*,\finpred^*}$ from Section \ref{sec:Fstar}.

\subsection{The extended function $F'$}
As in \cite{nova,othernova}, we first describe an extended function $F'$ that executes the folding verifier in addition to an iteration of $F$. Loosely speaking, extending $F$ into $F'$ is what enables bootstrapping a folding scheme into an IVC or RCG. We describe the function arguments first.\\

\noindent
$\instFprime=(z,\cnt,\acchash)$
\begin{itemize}
 \item $z$ - output for $F$
\item $\cnt$ - counter of IVC step
\item $\acchash$  - hash of accumulator
\end{itemize}
$\witFprime=(\instt,\insttbase,\instFprime_0,w,\prf)$
\begin{itemize}
\item$\instt$ - current accumulator instance
\item$\insttbase$ - previous accumulator instance
\item$\instFprime_0$ - instance (of $F'$) to be accumulated.
\item$w$ - private input for $F$
\item$\prf$ - proof for \protogal verifier
\end{itemize}


\noindent
$F'(\instFprime,\witFprime)=\accept$ if and only if:
\begin{enumerate}
    \item $F(\instFprime_0.z, w,z)=\accept$
\item If $\cnt>1$:
\begin{enumerate}
 \item  $\hash(\instt)=\acchash$.
\item $\instFprime_0.\acchash=\hash(\insttbase)$.
    \item $\instFprime_0.\cnt=\cnt-1$.
\item $\verpg(\insttbase, \instFprime_0, \prf, \instt )= \accept$.
    \end{enumerate}
    
\item If $\cnt=1$:
\begin{itemize}
    \item $\instFprime_0.z.\init=\true$

    \end{itemize}
\end{enumerate}

 Let $\rel,\relrand$ be relations for the \protogal version of the function $F'$ above.\footnote{In an updated version of the paper we will give more details on how to efficiently implement $F'$ as a \protogal relation.}\\
 
 
For readability, from now on we modify notational conventions and denote an accumulator by \acc, accumulator witness by \witacc, instance by \inst, and instance witness by \witinst. 
 \subsection{The relation \relfin}
We define the relation \relfin of pairs $(\instfin,\witfin)$ such that
\begin{itemize}
 \item $\instfin = (\acc,\recset,C,\zfin)$
 \item $\witfin = (\witacc,\accprev,\inst,\prf)$
\end{itemize}
such that
\begin{enumerate}
 \item $(\acc,\witacc)\in\relrand$.
\item $\verpg(\accprev,\inst,\prf,\acc)=\accept$.
\item $\hash(\accprev) = \inst.\acchash$.
\item $\inst.z=\zfin$.
\item $\inst.\cnt\leq C$.
\end{enumerate}

\noindent
Let $(\prvsnark,\versnark)$ be a zk-SNARK for \relfin.



\subsection{Main construction}\label{subsec:main}
We now describe the full prover and verifier for a given trivial RCG relation $\rel_{F,\finpred}$
Later we review how to use this to get a zk-SNARK for the relation \relexec of valid executions, given the reductions of previous sections. \\ \\
\noindent
\underline{$\prv(\inpF,\witF)$:}
\begin{enumerate}
 \item Let
$\inpF =(\zfin,C,\recset),\witF=(n,(z_0,\ldots,z_n),(w_1,\ldots,w_n))$. Recall that  $(\inpF,\witF)\in \rel_{F,\finpred}$ implies $F(z_{i-1},w_i,z_i)=\accept$ for each $i\in [n]$, $z_0.\init =\true, z_n=\zfin$ and $\finpred(\zfin,\recset) = \true$.
% \item For $i\in [n]$, let 


\item  Choose instance $\inst_0=(z_0,\cnt_0,\acchash_0)$ for arbitrary values $\cnt_0,\acchash_0$. Choose $\acc_0$ arbitrarily.
Let $\inst_1=(z_1,1,\acchash_1)$, $\witinst_1=(\acc_0,\acc_0, \inst_0, w_1, \prf_0)$ for arbitrary values $\acchash_1,\acc_0,\prf_0$. (We can choose some values arbitrarily as they aren't constrained in $F'$  when $\cnt=1$.)

\item Let $\acc_1$ be a randomly chosen satisfiable accumulator. Namley, $\acc_1=(\cm(\witacc_1),\betaa,e)$ where $\witacc_1$ and $\betaa$ are chosen randomly\footnote{We only need $\cm(\witacc_1)$ to be uniformly distributed over the image of \cm. According to \cm's structure, it may suffice to  choose only a small subset of $\witacc_1$'s coordinates randomly, and set the rest to zero.}, and $e$ is set to $e=\sum_{i\in [n]}\pow_i(\betaa)f_i(\witacc_1)$. 

\item For each $2\leq i\leq n$, compute
\begin{enumerate}
 \item $(\acc_i,\witacc_i,\prf_i) = \prvpg(\acc_{i-1},\witacc_{i-1},\inst_{i-1},\witinst_{i-1})$
\item $\inst_i = (z_i,i,\hash(\acc_i))$,
\item $\witinst_i=(\acc_i,\acc_{i-1},\inst_{i-1},w_{i},\prf_i)$
\end{enumerate}
\item
$(\acc,\witacc,\prf^*) = \prvpg(\acc_{n},\witacc_{n},\inst_{n},\witinst_{n})$.
\item Let $\prffin = \prvsnark(\instfin,\witfin)$ where
\begin{itemize}
 \item $\instfin = (\acc,\recset,C,\zfin)$
 \item $\witfin = (\witacc,\acc_n,\inst_n,\prf^*)$
\end{itemize}
\item Output $\prf=(\acc,\prffin)$.\\
\end{enumerate}

\noindent
\underline{$\ver(\inpF,\prf)$:}
\begin{enumerate}
 \item Parse \inpF as $(\zfin,C,\recset)$, $\prf$ as $(\acc,\prffin)$. 
 \item If $\finpred(\zfin,\recset)=\rej$ output \rej.
 \item Let $\instfin\defeq (\acc,\zfin,C,\recset)$.
 \item Return $\versnark(\instfin,\prffin)$.
\end{enumerate}

\begin{thm}\label{thm:main}
Let \relrcg be a trivial RCG relation. Assume that \hash is collision resistant, and that the assumptions in Theorem \ref{thm:pgsoundness} hold. Then
 $(\prv,\ver)$ is a zk-SNARK for $\rel_{F,\finpred}$. 
\end{thm}
\begin{proof}
 The main thing to prove is knowledge soundness. Let \adv be a recursive algebraic adversary.
 
We define the following extractor algorithm \ext:
\begin{enumerate}
 \item Given $(\instfin,\prf)$ - use the SNARK extractor to output $\witfin=(\witacc,\acc^*,\inst^*,\prf^*)$. 
 \item Let $n\defeq \inst^*.\cnt$. Denote $\acc_n\defeq \acc^*, \inst_n \defeq \inst^*$. 
 \item If $(\instfin,\witfin)\notin \relfin$ output \empt and abort. Otherwise $\acc_n,\inst_n\in K$ and let $\witacc_n,\witinst_n$ be the elements output by \adv such that $\cm(\witacc_n)=\acc_n$ and $\cm(\witinst_n)=\inst_n$.
 \item  For $i=n-1,\ldots ,1$:
 \begin{enumerate}
  \item If $(\inst_{i+1},\witinst_{i+1})\notin \rel$, output \empt and abort. Othewise, as $(\inst_{i+1},\witinst_{i+1})\in \rel$ implies  $\acc_{i},\inst_{i}\in K$, let $\witacc_{i},\witinst_{i}$ be the elements output by \adv such that $\acc_i= \cm(\witacc_{i}),\inst_i=\cm(\witinst_i)$.
  \item Parse $\inst_i$  as $(z_i,\cnt_i,\acchash_i)$. Parse $\witinst_i$ as $(\acc'_i,\acc_{i-1},\inst_{i-1},w_i,\prf_i)$. Note that through this parsing we have in particular defined $z_i,w_i$ and $\inst_{i-1}$. 
 \end{enumerate}
 \item Define $z_0\defeq \inst_0.z$.
\item Output $\witrcg\defeq ( n,(z_0,\ldots,z_n),(w_1,\ldots,w_n))$.
\end{enumerate}

Let $D$ be the event that \adv outputs $(\instrcg,\prf)$ such that $\ver(\instrcg,\prf)=\accept$.
If $D$ has probability \negl we are done. Otherwise, in order to prove knowledge soundness, we need to show that 
\[\prob\left( (\instrcg,\witrcg)\notin \relrcg|D\right)=\negl.\]
Asssume from now we are in the space conditioned on $D$.
Let $E_0$ be the event  $(\instfin,\witfin)\notin \relfin$.
For $i\in [n]$, let $E_i$ be the union of the following events:
\begin{enumerate}
 \item $(\inst_i,\witinst_i)\notin \rel$.
\item $(\acc_i,\witacc_i) \notin \rel$.
\item  $\hash(\acc_i)= \inst_i.\hash$.
\end{enumerate}
%  \item  $D_i:\acc'_i\neq \acc_i$.

We first argue that if $E_0,\ldots, E_n$  \emph{don't} occur, $(\instrcg,\witrcg)\in \relrcg$.
Denote $\instrcg = (\zfin,C,\recset)$ and $\instfin=(\acc,\instrcg)$.
Examinning the definition of $\relrcg$, the statement $(\instrcg,\witrcg)\in \relrcg$ is equivalent to the following conditions.
\begin{enumerate}
\item $\finpred(\zfin,\recset)=\true$:\ver checks this directly so we know it holds.
\item $\inst_n.z = \zfin$ and $n \leq C$: Follows from $( \instfin ,\witfin)\in \relfin$.
\item For all $i\in [n]$ $F(z_{i-1},w_i,z_i)=\accept$: Follows directly from $(\inst_i,\witinst_i)\in \rel$ for each $i\in [n]$. 
 \item $z_0.\init = \true$:
For this we first prove by backwards  induction on $i=n,n-1\ldots,1$ that $\inst_i.\cnt =i$.
By our definition of $\inst_n$ this holds for $i=n$. For the induction step, assume it holds for $i$.
Since $(\inst_i,\witinst_i)\in \rel$ we have $\inst_{i-1}.\cnt = \witinst_{i-1}.\inst.\cnt = \inst_i.\cnt -1 =i-1$.

In particular, $\inst_1.\cnt=1$ which implies when $(\inst_1,\witinst_1)\in \rel$ that $z_0.\init = \inst_0.z.\init=\true$.
\end{enumerate}

It is left to show that the probability of $E_0\cup E_1 \ldots \cup E_n$ is \negl.

For this purpose, we prove by backwards induction on $i$, for each $i\in [n]$, that
the probability that $E_i$ occurs given $E_0,E_{i+1},\ldots,E_n$ \emph{didn't} is \negl.


For $i=n$, assume $E_0$ didn't occur hence $(\instfin,\witfin)\notin \relfin$.
% from the knowledge soudness of $(\prvsnark,\versnark)$ e.w.p. \negl  $(\instfin,\witfin)\in \relfin$.
In this case $(\acc,\witacc)\in \relrand$ and $\verpg(\inst_n,\acc_n,\prf^*,\acc)=\accept$.
Define $\adv_n$ to be the algorithm that executes \ext and outputs $\acc_n,\inst_n,\acc,\witacc_n,\witinst_n,\witacc,\prf^*$.
From the knowledge soundness of \protogal applied to $\adv_n$ we have that the probability conditioned on $\neg E_0$ that
$E_n$ occured is \negl.

Now assume the induction hypothesis for some $i>1$. We have by the hypothesis that the event $D_i\defeq \neg(E_0\cup E_{i}\ldots\cup E_n)$ has probability $1-\negl$. Assume that we are in $D_i$. 
From $(\inst_{i},\witinst_{i})\in \rel$ we have that $\inst_{i}.\acchash = \hash(\acc'_{i})=\hash(\acc_i)$.
From the collision resistance of \hash, we thus have that the probability that $\acc_i\neq \acc_i$ is \negl.
If $\acc_i=\acc'_i$ we have from $(\inst_i,\witinst_i)\in \rel$  that $\ver(\acc_{i-1},\inst_{i-1},\prf_i,\acc_i) =\accept$.
Define $\adv_i$ to be the algorithm that executes \ext and outputs $\acc_{i-1},\inst_{i-1},\acc_{i-1},\witacc_{i-1},$ $\witinst_{i-1},\witacc_i,\prf_i$.
From the knowledge soundness of \protogal applied to $\adv_i$ we have that the probability that $(\inst_{i-1},\witinst_{i-1})\notin \rel$ or $(\acc_{i-1},\witacc_{i-1})\notin \relrand$ given $D_i$ is \negl. Noting that $(\inst_i,\witinst_i)\in \rel$ also implies $\hash(\acc_{i-1}) = \inst_{i-1}.\acchash$,  $\prob\left(E_{i-1} |D_i\right)=\negl$; which is the desired inductive statement.

\end{proof}
\subsection{Constructing proofs for \relexec}
Going through the reductions of previous sections, we get the following zk-SNARK for the relation \relexec from Section \ref{sec:validexec} describing valid executions.\\ \\
\noindent
\underline{$\prv(\instexec,\witexec)$:}
\begin{enumerate}
 \item Let $\instexec=(\root,C,\recset),\zfin=(\empty,\root,\false)$ and $\instrcg=(\zfin,C,\recset)$.
Let $\varphi$ be the map in Lemma \ref{lem:execasRCG}, and compute $\witrcg=\varphi(\witexec)$. 
\item Let $\witrcg = (n,z,w,s)$. As in Lemma \ref{lem:FtoFstar} define for each $i\in [n]$ $m_i\in \F^k$ such that $m_{i,j}$ is the number of times a record is when $s_{i,j}$ is an \add operation, and zero otherwise.
appears in $s_i$. 
\item Compute $h_n=\cm'(s,m)$, where $\cm'$ is as in Lemma \ref{lem:FtoFstar}. Compute
$(\alpha,\beta,\eps) = (\hash(\inchash,\recset,1),\hash(\inchash,\recset,2),\hash(\inchash,\recset,3)$.
\item Define $\zfin^*=(\zfin,h_n,\alpha,\beta,\eps),  \instrcg^*=(\zfin^*,C,\recset)$. Let $\witrcg^*=\varphi^*(\witrcg)$ where
$\varphi^*$ is the mapping from Lemma \ref{lem:FtoFstar}.
\item Compute the proof $\pi^*$ for $(\instrcg^*,\witrcg^*)\in \rel_{F^*,\finpred^*}$ according to \prv in Section \ref{subsec:main}.
\item Output $\pi=( (h_n,\alpha,\beta,\eps),\pi)$.
\end{enumerate}

\noindent
\underline{$\ver(\instexec,\pi)$:}
\begin{enumerate}
\item Parse  $\pi$ as $(\inchash,\alpha,\beta,\eps,\pi^*)$. Parse $\instexec$ as $(\zfin,C,\recset)$. Define
$\instrcg=( (\zfin,\inchash,\alpha,\beta,\eps),C,\recset)$.
 \item Let $\ver'$ be \ver from Section \ref{subsec:main}. Return $\ver'(\instrcg,\pi^*)$.
\end{enumerate}



\section{General Final predicate}

Sketch:
If we can end the IVC with a commitment $S$ to the concatenation of the \set{S_i} from all iterations. Then we can modify the final zk-SNARK in the last section to also check
that $\finpred(S_1,\ldots,S_n)=\accept$.

For this purpose, instead of the $F^*$ in section \ref{sec:Fstar} we add protogalaxy constraints to obtain a commitment of the concatenation of the previous
one with the $S_i$ of this iteration.

\subsection{Efficient Proofs of Concatenation with ProtoGalaxy}
% WIP
% TODO not sure if variable names conflict, check
Given a stack consisting of vectors of field elements of length $k$ and a total bound on the size of the stack $B$, we can define a system of linear protogalaxy constraints that a concatenation was correctly carried out.
We encode a vector $S_i$ of length $k i$ as a sequence of elements padded with zeros out to the total maximum vector length $k B$.
Then, to append a new vector of $k$ elements $\vec{a}$ to construct $S_{i+1} = \vec{a} \| S_i$ it is sufficient to check for all indices $j \in [k B]$ that
\begin{equation*} 
f_j(S_i, \vec{a}, S_{i+1}) = \begin{cases}
    a'_j - b_j & i \leq k \\
    a'_j - a_{j - k} & j > k
\end{cases}.
\end{equation*}

These constraints clearly encode the concatenation relation, and are well defined so long as $S_{i+1}$ does not contain $B$ elements.
If the stack is folded in order, then even in the new randomized instance, the zero padding is preserved.
That is, higher order elements above $i k$ are still zero after folding.
So prover only needs to perform one field multiplication per stack element to compute the folded witness.
These constraints are also linear so they do not contribute to any of the error terms in the $G(X)$ polynomial.

For all $j > k (i+1)$ notice that $f_j(S_i, \vec{b}, S_{i+1}) = 0$ identically, so they do not even contribute to the cost of computing the $F(X)$ polynomial.
Assuming all of these constraints are organized adjacently in the highest order indices of the overall protogalaxy constraint system, there is a straightforward modification of the $F(X)$ computation which incurs no cost for constraints that are identically zero.
In brief, if an entire subtree consists of leaves that are zero, then the prover can avoid any computation for that subtree.
Therefore, the total prover computation depends only on the size of the vector $S_{i+1}$.


% TODO is it outside the scope of the section to discuss how commitments to the vectors work?
% like the vector a should be the a' of the previous folding and we should copy the commitments
% then checking depends only on the instance.

\section*{Acknowledgements}
\bibliographystyle{alpha}
\bibliography{references}
\end{document}


